% Options for packages loaded elsewhere
\PassOptionsToPackage{unicode}{hyperref}
\PassOptionsToPackage{hyphens}{url}
%
\documentclass[
]{article}
\usepackage{amsmath,amssymb}
\usepackage{lmodern}
\usepackage{iftex}
\ifPDFTeX
  \usepackage[T1]{fontenc}
  \usepackage[utf8]{inputenc}
  \usepackage{textcomp} % provide euro and other symbols
\else % if luatex or xetex
  \usepackage{unicode-math}
  \defaultfontfeatures{Scale=MatchLowercase}
  \defaultfontfeatures[\rmfamily]{Ligatures=TeX,Scale=1}
\fi
% Use upquote if available, for straight quotes in verbatim environments
\IfFileExists{upquote.sty}{\usepackage{upquote}}{}
\IfFileExists{microtype.sty}{% use microtype if available
  \usepackage[]{microtype}
  \UseMicrotypeSet[protrusion]{basicmath} % disable protrusion for tt fonts
}{}
\makeatletter
\@ifundefined{KOMAClassName}{% if non-KOMA class
  \IfFileExists{parskip.sty}{%
    \usepackage{parskip}
  }{% else
    \setlength{\parindent}{0pt}
    \setlength{\parskip}{6pt plus 2pt minus 1pt}}
}{% if KOMA class
  \KOMAoptions{parskip=half}}
\makeatother
\usepackage{xcolor}
\usepackage[margin=1in]{geometry}
\usepackage{graphicx}
\makeatletter
\def\maxwidth{\ifdim\Gin@nat@width>\linewidth\linewidth\else\Gin@nat@width\fi}
\def\maxheight{\ifdim\Gin@nat@height>\textheight\textheight\else\Gin@nat@height\fi}
\makeatother
% Scale images if necessary, so that they will not overflow the page
% margins by default, and it is still possible to overwrite the defaults
% using explicit options in \includegraphics[width, height, ...]{}
\setkeys{Gin}{width=\maxwidth,height=\maxheight,keepaspectratio}
% Set default figure placement to htbp
\makeatletter
\def\fps@figure{htbp}
\makeatother
\setlength{\emergencystretch}{3em} % prevent overfull lines
\providecommand{\tightlist}{%
  \setlength{\itemsep}{0pt}\setlength{\parskip}{0pt}}
\setcounter{secnumdepth}{-\maxdimen} % remove section numbering
\ifLuaTeX
  \usepackage{selnolig}  % disable illegal ligatures
\fi
\IfFileExists{bookmark.sty}{\usepackage{bookmark}}{\usepackage{hyperref}}
\IfFileExists{xurl.sty}{\usepackage{xurl}}{} % add URL line breaks if available
\urlstyle{same} % disable monospaced font for URLs
\hypersetup{
  hidelinks,
  pdfcreator={LaTeX via pandoc}}

\author{}
\date{\vspace{-2.5em}}

\begin{document}

\hypertarget{figure-legend}{%
\section{Figure legend}\label{figure-legend}}

\hypertarget{figure-1-6}{%
\subsection{Figure 1-6}\label{figure-1-6}}

   \textbf{Fig. 1 \textbar{} Data stream of MCnebula workflow.} The
MCnebula workflow can be divided into two parts, depending on the
platform on which the data presents. The first is the part beyond R
(before MCnebula2): from the \textbf{Samples} to \textbf{LC-MS/MS} to
obtain the raw data; the stage of \textbf{Convert raw data} is
implemented using the popular MSconvert tool derived from Proteowizard;
for \textbf{Feature detection}, users can implement it with any LC-MS
processing tool, such as MZmine, XCMS, OpenMS, etc.; then .mgf or other
file format of MS/MS spectra is imported into SIRIUS for computations.
The part inside R, MCnebula2 implements integrating data and creating
Nebulae within `mcnebula' object (see section of MCnebula2 algorithm in
article for details).

   \textbf{Fig. 2 \textbar{} Comparison of classifying of MCnebula with
benchmark method and Evaluation of identification accuracy of MCnebula}.
\textbf{a)} Fig. 2a illustrates the comparison of MCnebula and benchmark
method (GNPS) with three indicators: classified number, stability,
relative false rate. The classified number is calculated as average sum
number of classified compounds of the selected 19 chemical classes. The
stability is calculated as: \(S = (N_{origin} - N_{x}) / N_{origin}\)
(\(N_{origin}\) is the average sum number of origin dataset; \(N_{x}\)
is the average sum number of medium noise dataset or high noise
dataset). The relative false rate is calculated as:
\(R = 1 - (1 - F) \times (1 - S)\) (\(F\) is the absolute false rate;
\(S\) is the stability, i.e., the average - lost rate in stability
assessment). \textbf{b)} Fig. 2b illustrates a comparison of classified
number of MCnebula and benchmark method. When noise is added into
original dataset, some number of classified features are occurred
\textless{} 50, a cut-off (number \(\geq\) 50) is set to exclude these
classes from assessment. \textbf{c)} Fig. 2b illustrates the identified
accuracy of MCnebula. A cut-off (Tanimoto similarity \(\geq\) 0.5) is
set to get chemical structures of high matching score for evaluation.

   \textbf{Fig. 3 \textbar{} Tracing top `features' in Child-Nebulae of
serum metabolomics dataset.} According to the rankings of `features' by
statistic analysis, the top `features' are marked with different colors
in Child-Nebulae.

   \textbf{Fig. 4 \textbar{} In-depth visualization of Child-Nebulae of
`Acyl carnitines'} \textbf{a)} and \textbf{b)}, refer to Fig. 3 and Fig.
S3. \textbf{c)} The nodes of top `features' are marked with color. The
nodes of `features' are annotated with structures, ring diagram and bar
plot of posterior probability of classes prediction (PPCP). The top
candidate of Chemical structure of `features' are mapped into nodes. The
Ring diagram map relative summed peak area of per `feature' detected
within each metadata group (NN: non-hospital, non-infected; HN:
hospital, non-infected; HS: hospital, survival; HM: hospital,
mortality). The nodes without ring diagram indicate `features' with
missing quantification value (these `features' were detected in our
re-analysis, but not in Wozniak et al.) The Bar plot map PPCP of
structural (sub-structural or dominant structural) classes for the
`feature'. \textbf{d)} The zoom in node of `feature' 2068 (ID) and its
legend.

   \textbf{Fig. 5 \textbar{} Heat maps of `Acyl carnitines' (ACs),
`Lysophosphatidylcholines' (LPCs), `Bile acids, alcohols and
derivatives' (BAs) in serum metabolomics dataset.} Figure 5\textbf{a},
\textbf{c} and \textbf{e} show heat maps of level of ACs, LPCs and BAs.

   \textbf{Fig. 6 \textbar{} Mechanism for filtering chemical classes of
MCnebula2.} This figure illustrates how MCnebula2 filters chemical
classes of prediction from `features' to form a Nebula-Index to create
Child-Nebulae. The \textbf{Inner filter} filter out the chemical classes
by Regex match of names (names without without Arabic numerals) and set
threshold for value of posterior probability. To create \textbf{Stardust
Classes}, the previous filtered data is re-grouped according to chemical
classes instead of `features' ID. The \textbf{Cross filter} conduct
further filtering of chemical classes via combining Stardust Classes and
`features' annotation data. The details of Cross filter was described in
MCnebula2 Algorithm section in article.

\hypertarget{supplementation}{%
\subsection{Supplementation}\label{supplementation}}

   \textbf{Fig. S1 \textbar{} Parent-Nebula of serum metabolomics
dataset.} In \textbf{Parent-nebula}, `features' are mapped as nodes in
network graph. The edges illustrated the spectral similarity of adjacent
`features'. Not all `features' are shown in the Parent-Nebula, as the
isolated nodes are removed.

   \textbf{Fig. S2 \textbar{} Child-Nebulae of serum metabolomics
dataset.} The Child-Nebulae are created according to chemical classes in
Nebula-Index. The classified `features' of chemical classes are mapped
into corresponding Child-Nebulae.

   \textbf{Fig. S3 \textbar{} Showing log\textsubscript{2}(Fold change)
of groups in Child-Nebulae of serum metabolomics dataset.} The
log\textsubscript{2}(Fold change) value of HM versus HS group is shown
in Child-Nebulae as gradient color. The nodes with white color indicate
`features' with missing quantification value (these `features' were
detected in our re-analysis, but not in Wozniak et al.)

   \textbf{Fig. S4 \textbar{} In-depth visualization of Child-Nebulae of
`Lysophosphatidylcholines' and `Bile acids, alcohols and derivatives'.}
See Fig. 4 for description.

   \textbf{Fig. S5 \textbar{} Pathway analysis of `Acyl carnitines'
(ACs), `Lysophosphatidylcholines' (LPCs), `Bile acids, alcohols and
derivatives' (BAs) in serum metabolomics dataset.} \textbf{a)} The
carnitine system in mitochondria. Abbreviation: CPT1,
carnitine-palmitoyltransferase-1; CACT, carnitine-acylcarnitine
translocase; CrAT, carnitine acetyltransferase; CPT2,
carnitine-palmitoyltransferase-2. \textbf{b)} Enrichment analysis of
LPCs in pagerank algorithm with Kyoto Encyclopedia of Genes and Genomes
(KEGG) metabolomic pathway. Abbreviation: P A2, phospholipase A2;
PC-Sterol O-AT, phosphatidylcholine-sterol O-acyltransferase; LP,
lysophospholipase; 1-AGPC O-AT, 1-acylglycerophosphocholine
O-acyltransferase; \textbf{c)} Enrichment analysis of BAs in pagerank
algorithm with KEGG metabolomic pathway. Abbreviation: βGC,
beta-glucuronidase; βGCS, beta-D-Glucuronoside; GT,
glucuronosyltransferase; TCDC 6α-H, taurochenodeoxycholate
6alpha-hydroxylase; TCDC, taurochenodeoxycholate; GCC, Glycocholate;
GCCDC, Glycochenodeoxycholate; Conju. BAs syn., `Conjugated bile acid
biosynthesis, cholate'; BA-CoA, bile acid-CoA:amino acid
N-acyltransferase.

   \textbf{Fig. S6 \textbar{} Tracing top `features' in Child-Nebulae of
herbal medicine dataset} According to the rankings of `features' by
statistic analysis, the top `features' are marked with different colors
in Child-Nebulae.

   \textbf{Fig. S7 \textbar{} MS/MS spectra of top `features' of herbal
medicine dataset.} For top `features', the mirrored MS/MS spectra plots
illustrated the raw MS/MS spectra (back bar) and the noise filtered
MS/MS spectra (red bar) predicted by SIRIUS. The dot above the bar
implied a corresponding relation. The top candidate of chemical
structure of `features' are mapped into the plot.

   \textbf{Fig. S8 \textbar{} Extracted ions chromatography (EIC) of top
`features' of herbal medicine dataset.} The EIC plot illustrates the
peak shape of the top `features' (drew via MCnebula; detected via
Automated Data Analysis Pipeline (ADAP) algorithm in MZmine2).

   \textbf{Fig. S9 \textbar{} Focus on location of top `features' in
annotated Child-Nebulae.} Fig. S9 \textbf{a}, \textbf{b} and \textbf{c}
illustrate local view of the annotated Child-Nebula of `Iridoids and
derivatives', `Dialkyl ethers' or `Phenylpropanoids and polyketides',
respectively. Fig. S9 \textbf{d} and \textbf{e} show the chemical
structure of `feature' (compound) of ID 2110 and ID 854, respectively.
Fig. S9 \textbf{f} and \textbf{g} show the mirrored MS/MS spectra (refer
to Fig. S7 for description) and extracted ions chromatography (EIC) of
the `features'.

\end{document}
