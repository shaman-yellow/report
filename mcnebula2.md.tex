\hypertarget{mcnebula-classified-visualization-for-spotlighting-structural-characteristics-of-underlying-biomarkers-and-unknown-compounds}{%
\section{\texorpdfstring{\textbf{MCnebula: Classified visualization for
spotlighting structural characteristics of underlying biomarkers and
unknown
compounds}}{MCnebula: Classified visualization for spotlighting structural characteristics of underlying biomarkers and unknown compounds}}\label{mcnebula-classified-visualization-for-spotlighting-structural-characteristics-of-underlying-biomarkers-and-unknown-compounds}}

\hypertarget{abstract}{%
\section{\texorpdfstring{\textbf{Abstract}}{Abstract}}\label{abstract}}

   Untargeted mass spectrometry is a robust tool for biological
research, but researchers universally time consumed by dataset parsing.
We developed MCnebula, a novel visualization strategy proposed with
multidimensional view, termed multi-chemical nebulae, involving in scope
of abundant classes, classification, structures, sub-structural
characteristics and fragmentation similarity. Many state-of-the-art
technologies and popular methods were incorporated in MCnebula workflow
to boost chemical discovery. Notably, MCnebula can be applied to explore
classification and structural characteristics of unknown compounds that
beyond the limitation of spectral library. Reference spectral data was
used for evaluation and MCnebula outperformed than popular benchmark
method in classify with high noise tolerance. In virtue of MCnebula, we
conducted investigation of human-derived dataset of serum metabolomics
by tracing structural classes of biomarkers so that facilitating
metabolic pathway spotlight. We also investigated a plant-derived
dataset of herbal \emph{E. ulmoides} to achieve a rapid identification
and explore chemical transformation during drug processing. MCnebula was
first integrated in R package and is now public available for custom R
statistical pipline analysis.

\hypertarget{introduction}{%
\section{\texorpdfstring{\textbf{Introduction}}{Introduction}}\label{introduction}}

   Analyzing untargeted liquid chromatography/tandem mass spectrometry
(LC-MS/MS) dataset is complicated, due to the massive of data volume,
complexity of spectrogram profiles, and structural diversity of
compounds. In the past decades, a great deal of research were anchored
to address the issues. Many technical software or web server attempted
to provide a one-stop bulk solution for data
analysis\textsuperscript{\protect\hyperlink{ref-2020p}{1}--\protect\hyperlink{ref-2016a}{4}}.
These solution apply or suggest flexible mass spectrogram processing
tools or analogous
algorithms\textsuperscript{\protect\hyperlink{ref-2012d}{5}--\protect\hyperlink{ref-2010}{8}}.
To reduce false-positive and false-negtive results, more algorithms have
been implemented for peak deconvolution, feature selection or
statistical
filtering\textsuperscript{\protect\hyperlink{ref-2017f}{9}--\protect\hyperlink{ref-2017i}{13}}.
Per feature corresponding to compound within sample or parallel samples,
and it prevalently equipped with fragmentation spectra to
identification. In this context, researchers have to be confronted with
a barrier: how to identify the compounds accurately?

   Until today, several strategies were developed for identifying with
fragmentation spectra, including \textbf{1)} Spectral library matching.
A number of public available databases were built to settle that via
achieving re-usability of reference fragmentation spectra, such as
MassBank, MassBank of \href{https://mona.fiehnlab.ucdavis.edu/}{North
America (MoNA)}, Global Natural Products Society molecular networking
(GNPS)\textsuperscript{\protect\hyperlink{ref-2016a}{4},\protect\hyperlink{ref-2010p}{14}--\protect\hyperlink{ref-2020cp}{17}}.
In counterpart, these fragmentation spectra are available via their web
severs, third-party platform (e.g.,
\href{http://prime.psc.riken.jp/compms/msdial/main.html\#MSP\%3E}{CompMass})
or specific tools
(MASST)\textsuperscript{\protect\hyperlink{ref-2020cm}{18}}. However,
comparing with structure database (PubChem harbours over 100 million
records), spectra library is too small in size that limit the
application of mass spectrometry. To cross this barrier, \textbf{2)}
Matching with fragmentation spectra of \emph{in silico} simulation.
\emph{In silico} tools have been increasingly developed for simulating
fragmentation
spectra\textsuperscript{\protect\hyperlink{ref-2010c}{19}--\protect\hyperlink{ref-2020cn}{27}}.
Some databases such as MoNA collated \emph{in silico} fragmentation
spectra for public
available\textsuperscript{\protect\hyperlink{ref-2013w}{28},\protect\hyperlink{ref-2015aj}{29}}.
\textbf{3)} \emph{In silico} prediction with matching learning.
Prevalently, the algorithms make machine train from reference mass
dataset or libraries, then `learned' how to predict chemical
fingerprints or principles so as to retrieve the correct structure
within structure
database\textsuperscript{\protect\hyperlink{ref-2012ab}{30}--\protect\hyperlink{ref-2021cy}{35}}.

   \emph{In silico} methods are developing quickly. Up to now, the
cutting-edge technology, SIRIUS
4\textsuperscript{\protect\hyperlink{ref-2019}{36}}, integrated with
many advanced algorithms of artificial intelligence, has been reported
accuracy rate reached 70\% while retrieving in structure libraries. This
method help to identify metabolites beyond the scope of spectra library.
While \emph{in silico} tools boost chemical identification, it is still
lack of an approach that incorporating and leveraging the
state-of-the-art technology into biological research, i.e.~biomarker
discovery in untargeted mass spectral dataset. Identification and
screening of biomarkers manually are time-consuming and the results are
impressed by subjective factors. In terms of identification, molecular
networking is increasingly popular due to its visualization and data
transparency\textsuperscript{\protect\hyperlink{ref-2016a}{4},\protect\hyperlink{ref-2020b}{37}}.
Thus, we proposed a preliminary idea, could clustering features for
visualization based on chemical classification contribute to biomarker
discovery or metabolic pathway spotlight?

   The history of classification in chemistry dates back to at least the
middle of the last century. The Chemical Fragmentation Coding system
developed by Derwent World Patent Index (DWPI) in 1963. Chemical
classification like Gene Ontology
(GO)\textsuperscript{\protect\hyperlink{ref-2000g}{38}}, has been
proposed, more systematically, organizing with taxonomy and ontology in
recent years\textsuperscript{\protect\hyperlink{ref-2016}{39}}.
ClassyFire is increasingly used for compound annotation either in mass
dataset analysis or not due to its computation available and
systematicness\textsuperscript{\protect\hyperlink{ref-2019bt}{40}--\protect\hyperlink{ref-2022al}{48}}.
The taxonomy and ontology for chemistry is beneficial. For example, a
hierarchical classification-based method, termed Qemistree, was proposed
to address chemical relationships at a dataset-wide
level\textsuperscript{\protect\hyperlink{ref-2021b}{45}}. Nevertheless,
we found that taxonomy or ontology for chemistry is not a one-off method
for pharmacological or biological researches. Numerous key metabolites
or drugs in classes are distributed in diverse hierarchies, such as
`Bile acids, alcohols and derivatives'
(subclass)\textsuperscript{\protect\hyperlink{ref-2020cr}{49}}, `Indoles
and derivatives'
(class)\textsuperscript{\protect\hyperlink{ref-2022am}{50}}, `Acyl
carnitines' (level
5)\textsuperscript{\protect\hyperlink{ref-2020s}{51}}. These classes
represent a family of compounds with either similar biological functions
or activity, however, function- or activity-independent scattered on
different branches of the diverse ranks of taxonomy. It confuse us and
which are potential biomarkers? Indeed, untargeted mass spectra dataset
is more like a black box. For unknown metabolites, locating the
appropriate classes as manifestation from a complicated list of chemical
taxonomy (\textgreater{} 4800 different categories) is challenging.
Previous study applied such analogous method while in binary comparison
but not yet
systematized\textsuperscript{\protect\hyperlink{ref-2021a}{52}}.

   For above comprehensive consideration, we proposed a classified
visualization method, named MCnebula, for untargeted LC-MS/MS dataset
analysis. MCnebula leverages the state-of-the-art \emph{in silico}
tools, SIRIUS workflow (SIRIUS, ZODIAC, CSI:fingerID,
CANOPUS)\textsuperscript{\protect\hyperlink{ref-2015a}{31},\protect\hyperlink{ref-2019}{36},\protect\hyperlink{ref-2021a}{52}--\protect\hyperlink{ref-2020a}{55}},
for compounds formulae prediction, structures retrieve and classes
prediction. For the first time, MCnebula integrates an abundance-based
classes selection algorithm for compounds annotation. MCnebula also
incorporates the benefits of molecular networking, i.e., intuitive
visualization and a great deal of information that can be conducted. In
virtue of MCnebula, we can switch from untargeted analysis to targeted
analysis which focusing on our interesting compounds or classes
precisely. MCnebula has massive potential functions, involving
metabolites identification, biomarker tracing in classes, drug
discovery, chemical change exploration, etc. In this article, two
datasets were applied to MCnebula in order to demonstrate the broad
utility of our method. One was a human-derived serum dataset that
correlated with mortality risk profiling of staphylococcus aureus
Bacteremia (SaB) The other was a plant-derived herbal dataset that
related to the processing of herbal medicine. We evaluated and validated
MCnebula with several datasets, involving reference spectra library and
published dataset.

\hypertarget{results}{%
\section{\texorpdfstring{\textbf{Results}}{Results}}\label{results}}

\hypertarget{overview-of-mcnebula}{%
\subsection{\texorpdfstring{\textbf{Overview of
MCnebula}}{Overview of MCnebula}}\label{overview-of-mcnebula}}

   MCnebula primarily performs an abundance-based class selection
algorithm before visualization. MCnebula tend to filter out those
classes with too large scope (e.g., possibly be `Lipids and lipid-like
molecules' but data dependent) or too sparse compounds (data depend). We
termed these summarised classes as nebula-index. To begin with, like
classical feature-based molecular networking (FBMN)
pattern\textsuperscript{\protect\hyperlink{ref-2020d}{56}}, features
make up the initial network, which we termed parent-nebula.
Subsequently, according to nebula-index and the posterior probability of
classes prediction (PPCP) of
features\textsuperscript{\protect\hyperlink{ref-2021a}{52}}, nodes or
edges from parent-nebula are divided into sub-networks. We termed these
sub-networks as child-nebulae and their names, termed nebula-name, are
in line with the classes name within nebulae-index (Fig.
{\protect\NoHyper\ref{fig:eu.couple.nebula}\protect\endNoHyper}). The
nebula-names contained the sub-structural or dominant-structural
characteristics for features within child-nebulae. Collectively, all the
network and sub-networks termed multi-chemical nebulae. In general,
parent-nebula is too informative to show, so child-nebulae was used to
dipict the abundant classes of metabolites in a grid panel, intuitively
(Fig. {\protect\NoHyper\ref{fig:eu.couple.nebula}\protect\endNoHyper}).
In a bird's eye view of child-nebulae, we can obtain many
characteristics of features, involving classes distribution, structure
identified accuracy, as well as spectral similarity within classes. An
end-to-end analysis using MCnebula is illustrated in Fig.
{\protect\NoHyper\ref{fig:workflow}\protect\endNoHyper}.

   Child-nebulae equipped with feature selection algorithm to trace
biomarker in classes (Fig.
{\protect\NoHyper\ref{fig:trace.bio}\protect\endNoHyper})\textsuperscript{\protect\hyperlink{ref-2017i}{13}}.
It assisted to focus on the targeted metabolites or compounds that we
were interested in from complicated untargeted analysis. Researchers can
focus on specific class based on priori knowledge. Additionaly, calling
nebula-name of interest in R with function, and a focused visualization
with in deep annotation is available (Fig.
{\protect\NoHyper\ref{fig:ac.zoom}\protect\endNoHyper}).

\hypertarget{evaluation-of-mcnebula}{%
\subsection{\texorpdfstring{\textbf{Evaluation of
MCnebula}}{Evaluation of MCnebula}}\label{evaluation-of-mcnebula}}

   \textbf{Classified accuracy.} For evaluation, we assigned GNPS
molecular networking as benchmark method, since its prestige, popularity
and as well emerged as a visualized
strategy\textsuperscript{\protect\hyperlink{ref-2016a}{4},\protect\hyperlink{ref-2020d}{56}}.
Considering parallelism and repeatability, we employed FBMN and equipped
it with MolnetEnhancer for assessment. In recent years, MolnetEnhancer
has extensively adopted to boost molecular networking function with
annotated
classification\textsuperscript{\protect\hyperlink{ref-2021cr}{57}--\protect\hyperlink{ref-2021cl}{64}}.
Although we attempted to compare both methods in completely parallel
way, there were several demarcation points for them: \textbf{1)}
MCnebula conducted abundance-based classes selection and filtering,
whereas benchmark method assigned all positive supperclass, class and
subclass annotation; \textbf{2)} MCnebula gather features into
classified index as child-nebulae, whereas in benchmark method, these
possibly be scattered across network or as isolated nodes. \textbf{3)}
MCnebula either performed dominant structural clustering or
sub-structural clustering, whereas for benchmark method, features were
annotated by dominant structural class.

   Both MCnebula and benchmark methods were run with a collection of
GNPS spectral library (positive ion mode). `Noise' was added into
spectra to evaluate the stability of both algorithm. For MCnebula, as
the figure shows (Fig.
{\protect\NoHyper\ref{fig:mc_noise_tolerance_bar}\protect\endNoHyper}),
the classified accuracy is around 80\% overall (`true' combined with
`latent', average 81.2\%, 80.6\%, 78.4\% in original dataset, middle
noise dataset and high noise dataset respectively). The annotated amount
exhibited high stability even with high noise.

   For benchmark method, we collated all the annotated classes
(superclass, class and subclass) and their harboured features. Some
classes were too sparse with features hence we set a cut off to filter
out those (features number \(\geq\) 50). The stats of three datasets
were gathered (Fig.
S\{\textsuperscript{\protect\hyperlink{ref-s.fig:molnet_noise_tolerance_bar}{\textbf{s.fig:molnet\_noise\_tolerance\_bar?}}}\}\{nolink=True\}).
With original dataset, the benchmark method exhibited high classified
accuracy and large annotated amount. For example, the figure (Fig.
S\{\textsuperscript{\protect\hyperlink{ref-s.fig:molnet_noise_tolerance_bar}{\textbf{s.fig:molnet\_noise\_tolerance\_bar?}}}\}\{nolink=True\})
showed nearly 500 `Flavonoides' classified and the accuracy was 87.0\%.
Nevertheless, the benchmark method has low tolerance to noise. When it
assessed with middle noise dataset and high noise dataset, the annotated
amount is reduced to 30\% and 60\%, respectively. Taking the same class
`Flavonoids' as an example, the annotated amount was decreased from 500
to nearly 200. In particular with high noise, the rest amount was only
around 100. The comparable classified amount of the in line classes are
showed in lollipop diagram (Fig.
{\protect\NoHyper\ref{fig:comp.accu_identi}\protect\endNoHyper}a). The
annotated amount of MCnebula exhibited better than benchmark method in
middle noise, high noise and even in original dataset for some classes.

   \textbf{Identified accuracy.} MCnebula leverages structures retrieved
by CSI:fingerID to annotate features within child-nebulae. Foremost,
CSI:fingerID retrieves through public available structure databases that
beyond the limitation of spectral library match. It facilitated
discovering of novel compounds. In current, the high identified accuracy
and outperforming of CSI:fingerID within SIRIUS workflow has been
reported\textsuperscript{\protect\hyperlink{ref-2019}{36},\protect\hyperlink{ref-2021}{65}}.
Herein, we evaluated the identified accuracy of features within
child-nebulae. The original dataset was employed for evaluation.
Overfitting issues did not exist since there was no spectral matching in
CSI:fingerID algorithm. Without any filter or exclusion, there were 8782
fragmentation spectra in original dataset. Few compounds were in
specific precursor adduct type, such as `{[}2M+H{]}+', `{[}2M+Na{]}+'
and even `{[}M+H-99{]}+' (totally 30 compounds). It is worth noting that
several compounds consists of Iodine element (totally 7 compounds) which
was sparse in current metabolite libraries. Considering untargeted
LC-MS/MS dataset as a complex ensembles of wild range of compounds, we
did no filter out. After preprocessing and collated by MCnebula, a total
of 8058 compounds were identified with formulae. Among them, a total of
6610 compounds were identified with chemical structures.

   For each feature in child-nebulae, we collated top score structure
for assessment. In line with classified evaluation, those
dominant-structural classes were picked. As figure shown (Fig.
{\protect\NoHyper\ref{fig:comp.accu_identi}\protect\endNoHyper}b), most
of the identified accuracy were between 60\% to 70\%. However, some
classes were quite low, such as `Long-chain fatty acids' (LCFA) or
`Lignans, neolignans and related compounds' (LN-RC). Actually,
researchers usually have no confidence for those structure with low
matching score. Tanimoto similarity provides the matching degree of
chemical fingerprints with
structures\textsuperscript{\protect\hyperlink{ref-2019}{36}}. In the
assessment, we set 0.5 as cut-off value with Tanimoto similarity to
filter those with low score.

\hypertarget{data-analysis-with-mcnebula}{%
\subsection{\texorpdfstring{\textbf{Data analysis with
MCnebula}}{Data analysis with MCnebula}}\label{data-analysis-with-mcnebula}}

   \textbf{Serum metabolic analysis.} To illustrate the application of
MCnebula in metabolism, we re-analyzed the serum data from Wozniak et
al.\textsuperscript{\protect\hyperlink{ref-2020s}{51}}. The serum
samples were collected from patients with \emph{Staphylococcus aureus}
bacteremia (SaB) or not and healthy volunteers. Overall, the samples
were divided into \textbf{1)} control groups, involving NN
(non-hospital, non-infected) and HN (hospital, non-infected);
\textbf{2)} infection groups, involving HS (hospital, survival), HM
(hospital, mortality).

   In previous research, a total of four top metabolites (TopMs) were
identified as 2-Hexadecanoylthio-1-Ethylphosphorylcholine (HEPC)
(original ID: 103 or 2385), sphingosine-1-phosphate (S1P) (original ID:
114), T4 (original ID: 1110) and decanoyl-carnitine (original ID: 119).
Except for HEPC, others were all identified in our re-analysis (Tab.
S\{\textsuperscript{\protect\hyperlink{ref-s.tbl:serum.bio}{\textbf{s.tbl:serum.bio?}}}\}\{nolink=True\}).
Intriguingly, `HEPC' was identified as
1-pentadecanoyl-sn-glycero-3-phosphocholine (LPC15:0) or its
stereoisomers. Indeed, HEPC and LPC15:0 are quite similar in structure,
but distinct in element constitution (corresponding to
C\textsubscript{23}H\textsubscript{48}NO\textsubscript{5}PS and
C\textsubscript{23}H\textsubscript{48}NO\textsubscript{7}P
respectively). They were clearly distinct in terms of chemical
classification. HEPC belong to `Cholines' (level 5) from `Organic
nitrogen compounds' (superclass) family, whereas LPC15:0 belong to
`Lysophosphatidylcholines' (level 5) from `Lipids and lipid-like
molecules' family. As a part of MCnebula workflow, sulfur element is
detectable for SIRIUS in isotopes pattern with high mass
accuracy\textsuperscript{\protect\hyperlink{ref-2009}{53}}. However, for
feature of original ID: 103 or 2385, there was no candidate formula that
harbouring sulfur element. In addition, the match of LPC15:0 was in high
COSMIC confidence score
(0.82)\textsuperscript{\protect\hyperlink{ref-2021}{65}}.

   In addition to four known metabolites, there were 11 new identified
metabolites and some metabolites were with high COSMIC confidence
(\textgreater{} 0.7) (Tab.
S\{\textsuperscript{\protect\hyperlink{ref-s.tbl:serum.bio}{\textbf{s.tbl:serum.bio?}}}\}\{nolink=True\}).
Child-nebulae were employed to trace those biomarker (TopMs) in abundant
classes. We set the child-nebula min possess as 10 features
(\(T_{min.possess} = 10\)), max possess percentage of all features as
0.1 (\(T_{max.possess.pct} = 0.1\)). To reduce hit classes, we
post-modified the max possess as 300 features (Fig.
{\protect\NoHyper\ref{fig:trace.bio}\protect\endNoHyper}). Overall, the
visualized child-nebulae covered 13 of 16 TopMs whereas the rest were
filtered out algorithmically. In-depth analysis of nebula-index, many
prominent classes were noteworthy for exploration.

\begin{itemize}
\item
  `Acyl carnitines' (ACs) were a sepsis related
  indicators\textsuperscript{\protect\hyperlink{ref-2018bc}{66}}, which
  as well agreed with Wozniak et al.~(Fig.
  S{\protect\NoHyper\ref{fig:ac.zoom}\protect\endNoHyper}). We verified
  5 identified ACs metabolites presented in previous article.
  Furthermore, more metabolites of ACs were identified in the
  child-nebula. As the ring diagrams show (the statistic data were
  merged from Wozniak et al.), most of the ACs were increasing in HM
  group. Comparing with the ACs in Top's or other previous identified
  ACs\textsuperscript{\protect\hyperlink{ref-2020s}{51}}, there were 4
  ACs with a more remarkable increasing (ID of 8795, 3286, 3203, 14196).
  Of note, these 4 ACs are out of the main large cluster of ACs, because
  their functional group locating at carbon chain end were all Carboxyl
  groups. MCnebula captured the same sub-structure precisely.
  Incidently, the attractive speculation was that the carboxy-modified
  ACs were more indicative while referring to sepsis and its liver
  dysfunction. The heat map of levels of ACs showed a correlation of
  their level with SaB infection (Fig.
  {\protect\NoHyper\ref{fig:path}\protect\endNoHyper}a). A fundamental
  role of ACs in tuning the switch between the glucose and fatty acid
  metabolism was
  reviewed\textsuperscript{\protect\hyperlink{ref-2018bi}{67}}. Their
  function implemented via bi-directional transport of acyl moieties
  Between cytosol and mitochondria (Fig.
  {\protect\NoHyper\ref{fig:path}\protect\endNoHyper}b).
\item
  `Lysophosphatidylcholines' (LPCs) were a group of bioactive lipids,
  which were not referred by Wozniak et al. In our re-analysis, three of
  TopMs were enriched in this class, involving LPC15:0 (ID: 1819,
  original ID: 103 or 2385). Indeed, LPCs were associated with septic
  mortality\textsuperscript{\protect\hyperlink{ref-2003n}{68},\protect\hyperlink{ref-2014ao}{69}}.
  Here, we found a correlation between LPCs level with SaB infection and
  mortality which implied a pathogenesis of sepsis. Focused on
  child-nebula of LPCs (Fig.
  S\{\textsuperscript{\protect\hyperlink{ref-s.fig:lpc_ba}{\textbf{s.fig:lpc\_ba?}}}\}\{nolink=True\}a),
  comparing with control groups, the level of some LPCs in infection
  groups was remarkably lower. The heat map of level of LPCs suggested a
  mortality risk of SaB infection, as the HM group was remarkably
  clustered (Fig. {\protect\NoHyper\ref{fig:path}\protect\endNoHyper}c).
  The significant LPCs (HS versus HM, \(p \lt 0.05\)) were performed
  with Kyoto Encyclopedia of Genes and Genomes (KEGG) metabolic pathway
  enrichment analysis. A kind of compounds termed
  `1-Acyl-sn-glycero-3-phosphocholine' (KEGG ID: C04230) were hit. The
  compounds of C04230 were characterized by its sub-structure (Fig.
  {\protect\NoHyper\ref{fig:path}\protect\endNoHyper}d). Almost all
  features classified in child-nebula of LPCs were belonging to C04230
  (Fig.
  S\{\textsuperscript{\protect\hyperlink{ref-s.fig:lpc_ba}{\textbf{s.fig:lpc\_ba?}}}\}\{nolink=True\}a).
  As the figure {\protect\NoHyper\ref{fig:path}\protect\endNoHyper}d
  showed, C04230 affected multiple downstream pathway and most of which
  were correlated with lipids metabolism.
\item
  `Bile acids, alcohols and derivatives' (BAs) act as an important
  signaling molecule associated with liver function and intestinal
  microbial homeostasis. Diverse BAs structures were discovered in the
  child-nebula (Fig.
  S\{\textsuperscript{\protect\hyperlink{ref-s.fig:lpc_ba}{\textbf{s.fig:lpc\_ba?}}}\}\{nolink=True\}b),
  and most were increasing in infection group. The heat map of level of
  BAs implied a high correlation of SaB infection. The significant BAs
  (control group versus infection group, \(p \lt 0.05\)) were performed
  with KEGG metabolic pathway enrichment analysis. The results foremost
  implied a correlation of SaB infection with bile secretion,
  cholesterol metabolism and primary bile acid biosynthesis.
\end{itemize}

   The above classes, together with steroids and fatty acids related
classes, all suggest a central role of liver in SaB induced infection or
mortality\textsuperscript{\protect\hyperlink{ref-2017au}{70}}. In
addition, the specific compound T4 in study of Wozniak et al.~were
clustered into mainly sub structural classes, `Phenylpropanoic acids'
and `Phenoxy compounds'. Unfortunately, as its uncommon element
constitution (Iodine), the classes failed to show the correlation with
other features. All the high quality predictions (Tanimoto similarity
\textgreater{} 0.5) of compounds in our re-analysis were collated
according to ClassyFire classification (Tab.
S\{\textsuperscript{\protect\hyperlink{ref-s.tbl:serum.compound}{\textbf{s.tbl:serum.compound?}}}\}\{nolink=True\}).

   \textbf{Herbal medicine analysis.} \emph{Eucommia ulmoides Oliv.}
(\emph{E. ulmoides})\textsuperscript{\protect\hyperlink{ref-2021n}{71}},
as a traditional Chinese medicine (TCM), after being processed with
saline water, was applied to the treatment of renal diseases for a long
time in China. Due to its complex composition, discovering chemical
changes during processing (such as processed with saline water) is
challenging. MCnebula was successfully applied to analyze plant-derived
chemical composition. Two kinds of processed \emph{E. ulmoides} dataset
were aligned in feature lists (Raw-Eucommia and Pro-Eucommia, before
processed with saline water or not), and run with MCnebula workflow.
Focused on abundant chemical classes, we set the nebula
\(T_{min.possess} = 30\) and \(T_{max.possess.pct} = 0.07\). Figure
{\protect\NoHyper\ref{fig:eu.couple.nebula}\protect\endNoHyper} shows
these focused classes. It was noticed there were several characteristic
classes `lignan', `iridoid' or `terpenoid'. In \emph{E. ulmoides},
numerous literatures have reported the medicinal value of compounds in
these
classes\textsuperscript{\protect\hyperlink{ref-2021cq}{72}--\protect\hyperlink{ref-2015q}{78}}.

   We filtered the features in child-nebulae by
\textbar{}\(log_{2}(FC)\)\textbar{} \textgreater{} 1 and ranked the
classes by variation relative abundance (Fig.
{\protect\NoHyper\ref{fig:eu.rank}\protect\endNoHyper}). The class of
`Pyranones and derivatives' (PDs) was in top one rank. Of note, many
Flavonoides were discovered in its nebula (Fig.
S\{\textsuperscript{\protect\hyperlink{ref-s.fig:pyran_iri}{\textbf{s.fig:pyran\_iri?}}}\}\{nolink=True\}a),
as its sub-structural Pyranone. Previous studies have reported
pharmacological functions of Flavonoides in \emph{E.
ulmoides}\textsuperscript{\protect\hyperlink{ref-2021cp}{79}--\protect\hyperlink{ref-2019x}{82}}.
Within the annotated nebula, the feature of ID:980 possess a remarkable
\(FC\). However, the structure was matched with low Tanimoto similarity
(0.38). Recur to overview child-nebulae, the level of some compounds
belonging `Lignan glycosides' (LG) and `Iridoid O-glycosides' (IOGs)
were changed with \textbar{}\(log_{2}(FC)\)\textbar{} \textgreater{} 1.
Strikingly, we found most of the IOGs were increasing in level (peak
area) after processing. Focused on IOGs, we visualized the annotated
child-nebula (Fig.
S\{\textsuperscript{\protect\hyperlink{ref-s.fig:pyran_iri}{\textbf{s.fig:pyran\_iri?}}}\}\{nolink=True\}b).
Structures of IOGs are similar in molecular skeleton, which contain a
sub-structural nucleus formed by a five-membered ring combined with a
six-membered ring. Among them, the feature with ID of 3918 is remarkably
increasing in Pro-Eucommia. Retrieving PubChem database via InChIKey
planar (first hash block of InChIKey), we found the briefest synonyms
termed `8,10-Didehydroargylioside', a compound without literature but
structural record. We showed extracted ion chromatography (EIC) and
fragmentation spectra of `8,10-Didehydroargylioside', together with some
related compounds of IOGs, LG or their parent classes (picked with
\textbar{}\(log_{2}(FC)\)\textbar{} \textgreater{} 1, Tanimoto
similarity \textgreater{} 0.5, and better peak shape) (Fig.
S\{\textsuperscript{\protect\hyperlink{ref-s.fig:eu.iso}{\textbf{s.fig:eu.iso?}}}\}\{nolink=True\}).
These compounds all showed remarkable alteration after processing. Among
them, `8,10-Didehydroargylioside' was a new generated compound after
processing. Via MCnebula and other chemical tools (ClassyFire, PubChem
database etc.), we identified a total of 582 compounds (Tab.
S\{\textsuperscript{\protect\hyperlink{ref-s.tbl:eu.compound}{\textbf{s.tbl:eu.compound?}}}\}\{nolink=True\}.
Some compounds were not reported before.

\hypertarget{discussion}{%
\section{\texorpdfstring{\textbf{Discussion}}{Discussion}}\label{discussion}}

   MCnebula is a novel visualization strategy that leverages
state-of-the-art \emph{in silico} technology and orients to overall
compounds within dataset. It offers a unique analytical perspective,
termed multi-chemical nebulae to achieve unknown compound identification
and classes focus. The visualization is equipped with a more precise,
flexible and perceptive capturing ability of chemical classes which is
different from classical molecular networking pattern. Meanwhile, it
draws on the superiority of the classical pattern. Recently, molecular
networking is a popular method for visualization and annotation of mass
spectra. Depending on fragmentation spectra similarity, structural
annotations are propagated via network-based
method\textsuperscript{\protect\hyperlink{ref-2012a}{83}--\protect\hyperlink{ref-2021d}{87}}.
Unfortunately, molecular networking is a highly spectral similarity
dependent method instead of compounds structural or classified
similarity. For example, Flavonoids were expected to be clustered
together as its specific class and structural similarity. However, in
previous research, it has been reported that some Flavonoids happened to
be absent from the cluster of many Flavonoids
compounds\textsuperscript{\protect\hyperlink{ref-2021a}{52}}. In this
context, visualization in a classified perspective is a better choice
for untargeted mass spectra dataset. Earlier in 2012, molecular
networking was proposed with visualization for mass data analysis for
the first time\textsuperscript{\protect\hyperlink{ref-2012a}{83}}. At
that time, \emph{in silico} tools for predicting compound classification
by fragmentation spectra were not available. Nowadays, with the
development of automatic classified \emph{in silico}
tools\textsuperscript{\protect\hyperlink{ref-2016}{39},\protect\hyperlink{ref-2021a}{52}},
it is time for a revolution of the visualization strategy.

   Herein, we evaluated MCnebula with its accuracy of both classify and
identification, it outperformed than feature-based molecular networking
(FBMN) equipped with MolnetEnhancer in classify. Particularly, MCnebula
exhibited higher tolerance of noise, since its workflow passed through
building fragmentation
tree\textsuperscript{\protect\hyperlink{ref-2015}{54}}. MCnebula is more
robust while dealing with metabolites of those without spectral library.
MCnebula leverages dataset of posterior probability of classes
prediction (PPCP) computed by CANOPUS to classify features and
facilitate annotation of even unknown metabolites.

   Untargeted metabolomics emerged to profile cellular and organismal
metabolism without prior knowledge
dependence\textsuperscript{\protect\hyperlink{ref-2016aq}{88},\protect\hyperlink{ref-2017at}{89}}.
Researchers in virtue of statistical methodologies from thousands of
features screen out biomarker, towards pharmaceutical, physiological or
pathological
mechanisms\textsuperscript{\protect\hyperlink{ref-2016ar}{90},\protect\hyperlink{ref-2016ao}{91}}.
These statistical approaches involved classical statistic and artificial
intelligence (e.g., random
forests)\textsuperscript{\protect\hyperlink{ref-2019bv}{92},\protect\hyperlink{ref-2021de}{93}}.
Both approaches were impossible to avoid specific biases, owing to the
complexity of feature set or algorithmic
stability\textsuperscript{\protect\hyperlink{ref-2017i}{13}}. Further,
evaluation in per feature level seemed incapable of profiling systematic
effects in
metabolites\textsuperscript{\protect\hyperlink{ref-2021a}{52}}. In this
view, analyzing at chemical classified level may be a comprehensive
settlement. However, we can not neglect the differences of metabolites
at the same classified level. For example, small-molecules belonging to
`Indoles and derivatives' harbour structural denpendent affection on
aryl hydrocarbon receptor
(AHR)\textsuperscript{\protect\hyperlink{ref-2019c}{94}}. Different
structural characteristics may lead to diverse activities. The
settlement for that is integrating either `per feature' level statistic
or classified level assessment. Therewith, MCnebula is proposed to
screen and trace biomarkers with higher confidence in classified level.

   MCnebula can be applied to discover biomarkers. We demonstrated the
application of MCnebula by re-analyzing serum metabolic dataset. The
accuracy of MCnebula for metabolite identification and its contribution
to the discovery of biomarkers in classified level was confirmed. We
found more `Acyl carnitines' (ACs) than previous study. Intriguingly, we
discovered additional classes, i.e.~`Lysophosphatidylcholines' (LPCs)
and `Bile acids, alcohols and derivatives' (BAs), that were not
concerned in previous study. Previously, LPCs have been extensively
investigated in the context of inflammation and atherosclerosis
development\textsuperscript{\protect\hyperlink{ref-2003n}{68},\protect\hyperlink{ref-2014ao}{69},\protect\hyperlink{ref-2020cv}{95},\protect\hyperlink{ref-2016at}{96}}.
In a recent review\textsuperscript{\protect\hyperlink{ref-2020cv}{95}},
the complex roles of LPCs in vascular inflammation have been well
described, involving the context-dependent pro- or anti-inflammatory
action, impact in innate immune cells and adaptive immune system, etc.
Decreasing level of LPCs was associated with wild range of diseases of
increasing mortality
risk\textsuperscript{\protect\hyperlink{ref-2016at}{96}}. The
investigation of spesis indicated LPCs concentrations in blood were
established correlation with severe sepsis or septic
shock\textsuperscript{\protect\hyperlink{ref-2014ao}{69}}; In addition,
LPCs was reported inversely correlate with mortality in sepsis
patients\textsuperscript{\protect\hyperlink{ref-2003n}{68}}. BAs'
disorder implied a liver dysfunction and imbalance of intestinal
microphylactic
homeostasis\textsuperscript{\protect\hyperlink{ref-2021dg}{97}}. The
chemical multiversity of BAs, which were discovered in the BAs'
child-nebula, were determined by the intestinal microbiome and allowed
for a complex regulation of adaptive responses in host. In our study,
the level of BAs showed higher correlation with SaB infection than ACs.
The decreased level of LPCs suggested a mortality risk of SaB infection.
From LPCs to BAs, steroids related classes, `Lineolic acids and
derivatives', and other fatty acids related classes, showed that liver
played a central role in SaB infection and mortality. Liver X receptors
(LXRs) harboured pivotal roles in the transcriptional control of lipid
metabolism\textsuperscript{\protect\hyperlink{ref-2018bd}{98}}. LXRs
modulate membrane phospholipid composition through activation of
lysophosphatidylcholine acyltransferase 3 (LPCAT3), which directly
related to LPCs\textsuperscript{\protect\hyperlink{ref-2021di}{99}}. In
addition, the above classes showed correlation with
LXRs\textsuperscript{\protect\hyperlink{ref-2018bd}{98}}. Unfortunately,
LXRs's specific role in SaB infection or mortality has not been
documented and beyond the scope of this research.

   In herbal dataset analysis, we showed a flexible exploration in
child-nebulae with a scope of classification. The instance was
enumerated with abundant classes. Child-nebulae could be set to trace
sparse classes according to manual definition. MCnebula is favorable to
compound identification even for unknown compound. For discovery of
novel compound from complex herbal medicine, the visualization of
child-nebulae is robust since it involved in scope of all abundant
classes, classification, structures and even sub-structural
characteristics. Although some specific database of plant-derived
compounds have been
constructed\textsuperscript{\protect\hyperlink{ref-2012ac}{15},\protect\hyperlink{ref-2015ak}{16}},
there were lack of enough fragmentation spectra for comprehensive
library match. In virtue of MCnebula, mostly via retrieving structural
libraries, a rapid and reliable resolution of complex compositions of
plant-derived can be achieved.

   In this article, due to limited space, few examples demonstrated
MCnebula's application. Indeed, MCnebula has a great potential in the
field of chemistry, pharmacy and medicine. The latter, beyond this
article, e.g., fields of application include natural products,
foodomics, environmental research etc. In addition, as an integrated
visualization method, MCnebula possibly be more popular with biologists
or chemists. Currently, MCnebula was first proposed and implemented in
the R language. In the future, its function and application will be
extensively expanded.

\hypertarget{methods}{%
\section{\texorpdfstring{\textbf{Methods}}{Methods}}\label{methods}}

\hypertarget{mcnebula-algorithm}{%
\subsection{\texorpdfstring{\textbf{MCnebula
algorithm}}{MCnebula algorithm}}\label{mcnebula-algorithm}}

   \textbf{Data preprocessing.} MCnebula algorithm builds on the feature
detection and SIRIUS compound identification workflow of untargeted
LC-MS/MS data. In brief, after feature detection of the untargeted
LC-MS/MS data (via MZmine2 or other mass data processing
tools\textsuperscript{\protect\hyperlink{ref-2020p}{1},\protect\hyperlink{ref-2016e}{6},\protect\hyperlink{ref-2006a}{7},\protect\hyperlink{ref-2010a}{100}}),
feature table and MS/MS list (.mgf format file) were exported; SIRIUS 4
soft, used .mgf file as input, performed SIRIUS (predict molecular
formula), ZODIAC (re-rank molecular formula), CSI:fingerID (retrieve
structure library), CANOPUS (predict compound classification) in
sequence. All results of SIRIUS workflow written down into SIRIUS soft
project space (a directory). The overview preprocessing step were as
follow:

\begin{itemize}
\tightlist
\item
  Convert raw mass spectrometry data (.RAW) to m/z extensible markup
  language (mzML) via MSconvert
  Proteowizard\textsuperscript{\protect\hyperlink{ref-2012d}{5},\protect\hyperlink{ref-2011b}{101}}.
\item
  Perform feature detection via MZmine2 (version 2.53).
\item
  Perform SIRIUS soft compound identification workflow, involving
  SIRIUS, ZODIAC, CSI:fingerID, CANOPUS.
\end{itemize}

   \textbf{MCnebula processing} MCnebula processing workflow was
implemented into a R package. In R console or studio, through loading
MCnebula package and employing several functions, MCnebula targeted at
SIRIUS soft project space, accomplished data collating, integrating and
visualization. The algorithms in detail were described:

\begin{itemize}
\item
  Collate molecular formulae. For each feature, as computation results,
  multiple molecular formula candidates may exist. MCnebula took
  comprehensive consideration of ZODIAC and CSI:fingerID scores to get
  top excellent formula. If CSI:fingerID retrieved any structure
  candidate, in default setting, MCnebula took formula of top score
  structure preferentially. While there was no structure candidates,
  MCnebula took the formula with top ZODIAC scores. Prioritization of
  picking top formula with either ZODIAC score or CSI:fingerID score can
  be reversed manually. Note that picking a correct molecular formula is
  the most essential step before structure identification as well as
  MCnebula workflow. Subsequently, The picked top formula for all
  features were gathered as MCnebula formula set (.MCn.formula\_set).
\item
  Collate structures. According to .MCn.formula\_set, for each feature,
  only considering top formula, MCnebula took top CSI:fingerID score
  structure within the candidates. Then the picked structures were
  gathered as MCnebula structure set (.MCn.structure\_set).
\item
  Collate PPCP. Analogously, according to .MCn.formula\_set, for each
  feature, only considering top formula, MCnebula took PPCP data of all
  classification. These data were gathered as MCnebula PPCP dataset
  (.MCn.ppcp\_dataset).
\item
  Summarise nebula-class. Within .MCn.ppcp\_dataset, for each feature,
  posterior probability of thousands of classes prediction exist. A
  threshold (\(T_{ppcp} = 0.5\) in default) was set to filter these
  data. Further, a paramater of classes hierarchy priority
  (\(P_{hierarchy.priority} = c(6, 5, 4, 3)\) in default, which are
  equivalent to level 5, subclass, class, superclass of ClassyFire) was
  set to filter and sort these classes. The raw .MCn.ppcp\_dataset
  contained a large amount of sub-sturcture or dominant structure
  classes prediction data. This step aimed to get those classes
  favorable for identification. After filtering, the dataset was
  gathered as MCnebula nebula class (.MCn.nebula\_class).
\item
  Summarise nebula-index. Although the raw .MCn.ppcp\_dataset was
  filtered via the previous step, all these classes were still too
  redundancy to perform an overview visualization of untargeted LC-MS
  dataset in classification. In this step, several measures were adopted
  to implement an auto-filtering.

  \begin{enumerate}
  \def\labelenumi{\arabic{enumi}.}
  \tightlist
  \item
    Those subtle classes which represented a location of chemical
    function were removed. In deed, MS/MS spectra is not proficient at
    distinguishing position isomerism. Due as characteristics of
    International uion AppliedChemistry Rules (IUPAC rules), this
    measures was achieved simply via removing those class names
    involving Arabic numerals while matching in pattern. For example, as
    sub-catalogues of `Hydroxyflavonoids', `6-hydroxyflavonoids' and
    `7-hydroxyflavonoids' were removed.
  \item
    Filter via features of max possess and min possess setting of a
    class. Using previous filtered classes to traverse
    .MCn.ppcp\_dataset. For those any class, while the PPCP of a feature
    reached \(T_{ppcp}\), the feature would be collated into the index
    of this class. After that, feature numbers in all classes index were
    stated respectively, and determined whether this class would be
    filtered out. The threshold of min possess (\(T_{min.possess}\)) was
    defined in absolute number whereas the threshold of max possess
    (\(T_{max.possess}\)) was defined in relative number (e.g., 20\% of
    all feature number). The former parameter aim at filtering out the
    class possess sparse features, and the later aim at filtering out
    the class which covering too large scope of compounds. For example,
    while overview \emph{E. ulmoides}, we appreciated abundant class
    such as `Lignan glycosides' instead of `Nitrobenzenes' etc.; for
    compounds belong to `Lipids and lipid-like molecules' (superclass),
    the better choices is representing them in that sub-catalogues, such
    as `Steroids and steroid derivatives', `Prenol lipids' etc.
  \item
    Filter out the classes that containing almost the same features. The
    threshold of top hierarchy (\(T_{iden.top.hierarchy} = 4\) in
    default. i.e., level of class in ClassyFire) and the threshold of
    identical factor (\(T_{iden.factor} = 0.7\) in default.) were
    defined. All classes lower than \(T_{iden.top.hierarchy}\) were
    compared in binary. While each other possess more than
    \(T_{iden.factor}\) of the same features, the class which possess
    less features would be filtered out. In deed, only few classes were
    filtered out in this algorithm. However, if a lower value of
    \(T_{iden.factor}\) is set, some sub-catalogues may be removed
    (e.g., `Hydroxyflavonoids' removed but `Flavonoids' kept).
  \item
    Filter out the classes of features with low degree of structural
    identification. In most of cases, incorrect molecular formula lead
    to failed fingerprint which predicted from corresponding
    fragmentation tree. Both structure and PPCP were matched or computed
    depending on fingerprint. The false positive molecular formula would
    cause both wrong in structure identification and classes prediction.
    Reflecting in class, some classes harboured abundant features,
    however, almost no structures were matched or all matched structures
    with low similarity score. To filter out those classes, a similarity
    score-based algorithm was defined. First, the evaluation of
    similarity score type was set (\(P_{simi.score}\) = `Tanimoto
    similarity' in default). Then a cut-off of similarity score
    (\(T_{simi.score} = 0.3\) in default) was set. All classes harboured
    less than min reached ratio (\(T_{min.reach} = 0.6\) in default) of
    eligible features were filtered out. Ultimately, the rest classes
    and affiliated features were gathered as MCnebula nebula index
    (.MCn.nebula\_index).
  \end{enumerate}
\item
  Generate parent-nebula. Analogous with molecular networking,
  parent-nebula consists of nodes (vehicles of feature information or
  annotation) and edges (annotation of fragmentation spectra similarity)
  data. To get edges and nodes data and merged as parent-nebula,
  MCnebula implement:

  \begin{enumerate}
  \def\labelenumi{\arabic{enumi}.}
  \tightlist
  \item
    Evaluation of MS/MS spectrum similarity among features. MCnebula
    integrated `compareSpectra' function of MSnbase R package to
    calculated cosine similarity (dotproduct) among MS/MS
    spectra\textsuperscript{\protect\hyperlink{ref-2020v}{102}}. Unlike
    popularly spectral comparing
    method\textsuperscript{\protect\hyperlink{ref-2020d}{56},\protect\hyperlink{ref-2012a}{83}},
    instead of using raw MS/MS spectra, MCnebula collated all noise
    filtered MS/MS spectra for comparison. The noise filtered spectra
    were acquired from SIRIUS project
    space\textsuperscript{\protect\hyperlink{ref-2015}{54},\protect\hyperlink{ref-2017}{103}}.
    Different molecular formulae candidates of one feature, the
    corresponding MS/MS spectra may assign with diverse `valid' or
    `noise' peak pattern. To in line with above algorithm, all spectra
    picking based on molecular formulae within .MCn.formula\_set. In
    addition, to reduce time-consuming computation, spectra similarities
    were calculated only within the same nebula-index
    (\(P_{iden.class}\)); only classes hierarchy equal to or lower than
    thereshold (\(T_{min.hierarchy} = 5\), in default, i.e., subclass of
    ClassyFire) were considered. Furthermore, if total feature number
    was more than 2000 (in default), ZODIAC scores
    (\(T_{min.zodiac} = 0.9\) in default) and Tanimoto similarity scores
    (\(T_{min.tanimoto} = 0.4\) in default) were utilized to exclude
    features from computation. After that, a threshold of edges
    (\(T_{edge.filter} = 0.3\) in default) was set to filter out low
    similarity. The results were formatted as edges data
    (.MCn.parent\_edges).
  \item
    Merging of multiple dataset. MCnebula merged .MCn.formula\_set with
    .MCn.structure\_set as nodes data (.MCn.parent\_nodes).
  \item
    Integration of .MCn.parent\_nodes and .MCn.parent\_edges as `graph'
    project of igraph R package
    (.MCn.parent\_graph)\textsuperscript{\protect\hyperlink{ref-2006b}{104}}.
    In addition, .grahml format file of parent-nebula was exported for
    supporting interactive exploration within
    Cytoscape\textsuperscript{\protect\hyperlink{ref-2003}{105}}.
  \end{enumerate}
\item
  Generate child-nebulae. Depending on .MCn.nebula\_index,
  .MCn.parent\_nodes and .MCn.parent\_edges were accordingly divided and
  gathered as a variety of `graph' project. At the meantime, for one
  nodes, a threshold of max possessing (\(T_{max.edges} = 5\) in
  default) was defined to reduce edges for better visualization of
  child-nebulae. The edges imply lower similarity were preferentially be
  cut off. In the end, all child-nebulae `graph' were saved into
  .MCn.child\_graph\_list and as well exported as .graphml format file,
  respectively. Note that a feature may exists in multiple
  child-nebulae, since compounds could be defined to diverse classes
  attributes to its sub-structures or dominant-structure.
\item
  Visualize parent-nebula and child-nebula. In this step, `graph' object
  obtained previously were visualized as individual or grid-based
  network via multiple R
  packages\textsuperscript{\protect\hyperlink{ref-2006b}{104},\protect\hyperlink{ref-2016g}{106}--\protect\hyperlink{ref-2020u}{114}}.
  In addition, Other R packages were used pass through all data
  processing if
  neccessary\textsuperscript{\protect\hyperlink{ref-2021x}{115}--\protect\hyperlink{ref-2022e}{120}}.
\end{itemize}

   R presents a variety of flexible scientific stating and graphing
tools. MCnebula provided a chanel to harbour massive of auto-annotated
data of SIRIUS workflow into R analysis pipeline. Users were encouraged
to leverage R tools facilitating data integration and parsing. The
visualization format of multi-chemical nebulae facilitate data
perspective, which is favorable both for compound identification and for
discovering biomarkers. In addition, GNPS FBMN could be incorporated
into MCnebula for analysis. MCnebula could take the `edge' file (a table
file) generated by FBMN and performs classified grid as multi-chemical
nebulae.

\hypertarget{mcnebula-evaluation}{%
\subsection{\texorpdfstring{\textbf{MCnebula
evaluation}}{MCnebula evaluation}}\label{mcnebula-evaluation}}

   \textbf{Spectra dataset for evaluation.} The spectra collection (in
positive ion mode, for more spectra data) of GNPS MS/MS library were
used for evaluation (.msp file)
(\url{http://prime.psc.riken.jp/compms/msdial/main.html\#MSP}). As
Fragmentation spectra in reference library generaly possess high
quality, and while used for evaluation of library match, it may caused
overfitting. To address the issue, refer to
ref.\textsuperscript{\protect\hyperlink{ref-2021}{65}}, we added `noise'
into these MS/MS spectra. In brief, the `noise' involves mass shift,
intensity shift, and the inserted noise peak; of note, the magnitude
factors for these shift were drawn randomly from function of normal
distribution. Overall, we simulated two model of `noise' (medium noise
and high noise). The `noise' simulation was achieved in custom R script.
The algorithm and parameters were parallel to the
ref.\textsuperscript{\protect\hyperlink{ref-2021}{65}}. We assign these
dataset as original data, middle noise data and high noise data.

   For another issue, the spectra collection did not possess isotopes
pattern. In real LC-MS processing (feature detection), isotope peaks
were grouped and merged, which favorable for SIRIUS to detect some
specific element\textsuperscript{\protect\hyperlink{ref-2009}{53}}. To
simulate isotopes pattern, we used function of `get.isotopes.pattern'
within `rcdk' R package to get isotope mass and its
abundance\textsuperscript{\protect\hyperlink{ref-2007j}{121}}. Further,
these mass were considered for the adduct type to increase or decrease
exact mass. For the `intensity' of these isotopes pattern, we simulated
as relative intensity, i.e., the abundance of isotopes multiply by 100
as the value. These `isotope peaks' were merged into MS1 list of its
compounds. All the spectra collections were formatted to fit with input
of MCnebula workflow or benchmark method (.mgf file and feature
quantified table).

   \textbf{Evaluation method.} The three simulated data were all run
with MCnebula workflow and benchmark method. While these data were put
into SIRIUS 4 command-line interface (CLI) (version 4.9.12) for
computation, the MS/MS spectra with empty fragmentation peak were
auto-filtered. In addition, to reduce computation time, the compounds
with over 800 m/z precursor were filtered out manually. These filtered
out compounds were excluded from ultimate accuracy stat. In this
context, several compounds harboured Iodine element were excluded from
stating either (totally 7), as it cost more time for SIRIUS to detect
that (it is sparse in metabolites, SIRIUS do not detect that in default
setting). There were 8782 MS/MS spectra within the raw collection, and
after filtered or excluded, totally 7829 compounds for ultimate
evaluation.

   The assessment of classification was in virtue of
ClassyFire\textsuperscript{\protect\hyperlink{ref-2016}{39}}. In detail,
we traversed the raw .msp spectra file to collate metadata of these
compounds, involving structure annotation. The International Chemical
Identifier Key (InChIKey) of these compounds were available for
ClassyFire database retrieve. However, since ClassyFire only support for
those chemical identity of which structure have been classified, we
noticed all the InChIKeys were vetoed. To address that, we employed
first hash block of these InChIKeys (InChIKey planar, represent
molecular skeleton) to touch PubChem application programming interface
(API)
(\url{https://pubchemdocs.ncbi.nlm.nih.gov/pug-rest})\textsuperscript{\protect\hyperlink{ref-2022ak}{122}}.
Accordingly, we got all the possibly InChIKeys of isomerism (stereo,
isotopic
substitution)\textsuperscript{\protect\hyperlink{ref-2012e}{123}}. The
classification of small molecules are depending on its molecular
skeleton hence these chemical identities possess the same InChIKey
planar are identical in classification. We pushed the obtained InChIKey
list to ClassyFire to catch classification. In R script, once any
InChIKey of isomerism meet the classified data in database, the
acquisition status for this molecular skeleton was turn off. In the end,
all these chemical annotation were collated, integrated and assigned as
standard reference.

   The discrepancies between the MCnebula and benchmark methods in terms
of algorithm and classified results disallow them to be evaluated at
completely the same level. First, we evaluated both methods
respectively. For MCnebula, before stat the accuracy, we interrogated
the child-nebulae generated from the raw spectral collection (Fig.
S\{\textsuperscript{\protect\hyperlink{ref-s.fig:collection.raw.child}{\textbf{s.fig:collection.raw.child?}}}\}\{nolink=True\}).
The child-nebulae at least in half were classified based on
sub-structural class, such as `Organic carbonic acids and derivatives',
`Hydroxy acids and derivatives'. These classes were small in structural
size and were chemical function group within compounds. The principle of
ClassyFire is selecting the most dominant structural class of compounds
to substitute\textsuperscript{\protect\hyperlink{ref-2016}{39}}.
However, in perspective of drug discovery, structure determines potency;
many pharmacological action possibly depends on these sub-structure. To
locate more universality among features, in algorithm, we reserved these
classes in nebula-index. Sub-structural classify for benchmark method
not available, hence we neglected these classes in evaluation. The rest
classes, nevertheless, still possibly be sub-structural class while meet
some compounds. We assigned three levels for evaluation, i.e., `true',
`latent', `false' (Fig.
{\protect\NoHyper\ref{fig:mc_noise_tolerance_bar}\protect\endNoHyper}).

   To assess the identification of classes or structures, the workflow
results were merge with standard reference by InChIKey planar. Once the
identification results are in line with standard reference, we assigned
it as `true'. For assessment of structure identification, indeed, such
strategy neglect stereochemistry.

\hypertarget{serum-dataset}{%
\subsection{\texorpdfstring{\textbf{Serum
dataset}}{Serum dataset}}\label{serum-dataset}}

   We re-analyzed 245 LC-MS/MS data (.mzML) from MASSIVE (id no.
\href{https://massive.ucsd.edu/ProteoSAFe/QueryMSV?id=MSV000079949}{MSV000083593})
(blanks, controls and
samples)\textsuperscript{\protect\hyperlink{ref-2020s}{51}}. MZmine2
(version 2.53) was performed for feature detection. The detection
workflow mainly involves \textbf{1)} Automated Data Analysis Pipeline
(ADAP) for peak detection and
deconvolution\textsuperscript{\protect\hyperlink{ref-2017f}{9}},
\textbf{2)} isotopes peak finder, \textbf{4)} parallel samples join
alignment, \textbf{5)} gap filling algorithm. While exporting MS/MS
spectra (.mgf) for SIRIUS 4 software computation, spectra were merged
across samples into one fragmentation list with 30\% Peak Count
threshold filtering. The feature detection workflow was refer to
\href{https://ccms-ucsd.github.io/GNPSDocumentation/featurebasedmolecularnetworking-with-mzmine2/}{FBMN
preprocessing} and
\href{https://boecker-lab.github.io/docs.sirius.github.io/prerequisites/}{SIRIUS
computational prerequisites}. The output .mgf was run with SIRIUS 4
software (version 4.9.12) for computation with
SIRIUS\textsuperscript{\protect\hyperlink{ref-2019}{36},\protect\hyperlink{ref-2009}{53}},
ZODIAC\textsuperscript{\protect\hyperlink{ref-2020a}{55}},
CSI:fingerID\textsuperscript{\protect\hyperlink{ref-2015a}{31}},
CANOPUS\textsuperscript{\protect\hyperlink{ref-2021a}{52}}. In
particular, SIRIUS was customized set to detect Iodine element while
predicting formula. The COSMIC confidence scores within SIRIUS 4
software output were used for assessment of
identification\textsuperscript{\protect\hyperlink{ref-2021c}{124}}.
MCnebula package were used for collating data from SIRIUS 4 output file
and visualizing child-nebulae or individual child-nebula.

   In research of Wozniak et al., an ensemble feature selection (EFS)
approach and Mann-Whitney U (MWU) tests were used for feature selection
(survival versus
mortality)\textsuperscript{\protect\hyperlink{ref-2017i}{13}}. We
collated the feature annotation (m/z and retention time) of the top 25
EFS metabolites and the top MWU metabolites which were identified
before, i.e.~thyroxine (T4) and decanoyl-carnitine. We named these
metabolites as TopMs. We aligned TopMs with our re-analyzed results by
m/z (0.01 tolerance) and retention time (0.3 min tolerance) (Tab.
S\{\textsuperscript{\protect\hyperlink{ref-s.tbl:serum.bio}{\textbf{s.tbl:serum.bio?}}}\}\{nolink=True\}).
A total of 16 TopMs were aligned. Due to the algorithmic difference in
feature detection, the re-analyzed feature list was not exactly in line
with that of before. It should be noted that the new identity number
(ID) was generated in this article, whereas original ID was from
previous study\textsuperscript{\protect\hyperlink{ref-2020s}{51}}. More
preferable augments for MCnebula workflow was set (e.g., at least 10 ppm
or 0.002 m/z tolerance for Automated Data Analysis Pipeline
(ADAP)\textsuperscript{\protect\hyperlink{ref-2017f}{9}}). The
inconsistency were neglected, since the algorithmic evaluation of
feature detection is out of the scope. For statistic data, we obtained
\href{https://www.cell.com/cms/10.1016/j.cell.2020.07.040/attachment/f13178d1-d1ee-4179-9d33-227a02e604f1/mmc3.xlsx}{Metabolomics
Data Resource} of Wozniak et al.~and aligned with our re-analyzed
feature list (.csv file export from MZmine2). All statistic data (peak
area) for serum dataset was used the
previous\textsuperscript{\protect\hyperlink{ref-2020s}{51}}.

   Kyoto Encyclopedia of Genes and Genomes (KEGG) metabolic pathway
enrichment analysis was performed with `Lysophosphatidylcholines' (LPCs)
and `Bile acids, alcohols and derivatives' (BAs), respectively. We used
the identified InChIKey planar of structures to hit compounds in
metabolic pathway. In detail, firstly, in order to avoid the identified
structural deviations due to stereoisomerism, the InChIKey planar were
used to obtain all possible InChIKeys via PubChem API. In this step,
PubChem CID of those compounds were also obtained. The R package of
MetaboAnalystR was used for converting PubChem CID to KEGG
ID\textsuperscript{\protect\hyperlink{ref-2020cx}{125}}. Many compounds
were not related to metabolic pathway so those were filtered out. The R
package of FELLA was used for KEGG enrichment with `pagerank'
algorithm\textsuperscript{\protect\hyperlink{ref-2018bj}{126},\protect\hyperlink{ref-ilprints422}{127}}.

\hypertarget{e.-ulmoides-dataset}{%
\subsection{\texorpdfstring{\textbf{\emph{E. ulmoides}
dataset}}{E. ulmoides dataset}}\label{e.-ulmoides-dataset}}

   \textbf{Material and processing.} \emph{E. ulmoides} dried bark was
obtained from company of ZheJiang ZuoLi Chinese Medical Pieces LTD.
Raw-Eucommia and Pro-Eucommia were prepared as following: (1)
Raw-Eucommia: The shreds or blocks of \emph{E. ulmoides} dried bark were
took, powdered and passed through 80-mesh sieves for further process.
(2) Pro-Eucommia: The shreds or blocks of \emph{E. ulmoides} dried bark
were took, fried with saline water (the amount of salt is 2\% of
\emph{E. ulmoides}, add 10 fold of water to dissolve), and smothered in
airtight for 30 min. Then, the barks were dried in oven at 60 °C,
followed by baking at 140 °C for 60 min. Finally, the baked barks were
powdered and passed through 80-mesh sieves for further process. The
processing method was based on previous studies of \emph{E.
ulmoides}\textsuperscript{\protect\hyperlink{ref-2010b}{128}}.

   \textbf{Sample preparation.} 2 g of Raw-Eucommia powder and
Pro-Eucommia powder were weighed, respectively, added 50 ml of
methanol/water (1:1, v/v) followed by ultrasonic (20 kHz for 40 min).
After ultrasonic, the mixture was filtered to obtain filtrate and
residue. The residue was added with 50 ml of methanol/water (1:1, v/v)
and extracted with ultrasonic (40 kHz, 250 W for 20 min) again. The
mixture was filtered. Then, the filtrate of the two extracts was
combined, the solvent was evaporated. Methanol/water (1:1, v/v) was
added to redissolve the extract and the volume was fixed to 5 ml.
Finally, the supernatant was obtained by centrifugation (12,000 r.p.m.
for 10 min) for further LC−MS analysis.

   \textbf{LC--MS experiments.} LC−MS analysis was performed using a
Dionex Ultimate 3000 UHPLC system (Dionex, Germany) coupled with a
high-resolution Fourier-transform mass spectrometer (Orbitrap Elite,
Thermo Fisher Scientific, Germany) using a Waters Acquity HSS T3 column
(1.8 μm, 100 mm \(\times\) 2.1 mm, Waters Corporation, Milford, MA,
USA). Solvent A, formic acid/water (0.1:99, v/v), and solvent B, formic
acid/acetonitrile (0.1:99, v/v), were used as the mobile phase. The
gradient profile for separation was as follows: 2\% of solvent B at
0min, 5\% of solvent B at 2 min, 15\% of solvent B at 10 min, 25\% of
solvent B at 15 min, 50\% of solvent B at 18 min, 100\% of solvent B at
23 min, 2\% of solvent B at 25 min, and 2\% of solvent B at 30 min. The
flow rate was 0.3 ml/min. The column temperature was set at 40°C. Mass
spectrometric analysis was performed using an Orbitrap Elite instrument
equipped with an ESI source (Thermo FisherScientific, Germany) that
operated in the negtive ionization mode. The ESI source was operated at
50 °C with a capillary temperature of 275 °C, an ionization voltage of
3.5 kV, and a sheath gas flow rate of 35 L/min. The survey scans were
obtained in the Orbitrap mass analyzer operating at a 120,000 (full
width at half-maximum) resolving power. A mass range of 100−1500 m/z and
a normalized collision energy of 30 eV were used for survey scans. The
analysis method was set to analyze the top 10 most intense ions from the
survey scan, and a dynamic exclusion was enabled for 15 s.

   \textbf{MCnebula Workflow.} E.ulmoides dataset were preprocessed with
MZmine2 for feature detection. The algorithmic workflow was similar to
serum metabolic dataset. Here, some peaks in bad shape were filtered out
manually. While export MS/MS spectra (.mgf) for SIRIUS computation,
spectra were merged across samples into one fragmentation list without
filtering. Similar to processing in serum dataset, MCnebula was used for
collating data from SIRIUS 4 output file and visualize child-nebulae or
individual child-nebula. Statistical analysis of nebula-index (classes)
with peak area data were conducted. Those classes with features of which
\textbar{}\(log_{2}(FC)\)\textbar{} \textgreater{} 1 (\(FC\): peak area
of Pro-Eucommia / Raw-Eucommia) were extracted as neo-nebula-index. The
variation relative abundance (for each classes, number of features
\textbar{}\(log_{2}(FC)\)\textbar{} \textgreater{} 1 divided by sum)
were stated and sorted as rank. Based on neo-nebula-index, we
re-visualized the overview child-nebulae and mask features with value of
\(log_{2}(FC)\) (Fig.
{\protect\NoHyper\ref{fig:eu.rank}\protect\endNoHyper}).

\hypertarget{data-processing}{%
\subsection{\texorpdfstring{\textbf{Data
processing}}{Data processing}}\label{data-processing}}

   Raw data (.raw) were converted to m/z extensible markup language
(mzML, i.e., .mzml format data) in centroid mode using MSConvert
ProteoWizard\textsuperscript{\protect\hyperlink{ref-2012d}{5},\protect\hyperlink{ref-2011b}{101}}.
For \emph{E. ulmoides} dataset, the .mzml files were processed with
MZmine2 (v.2.53) on Windows 10 1909 64-bits PC (Intel Core i5-8300H,
2.30 GHz, 16Gb of
RAM)\textsuperscript{\protect\hyperlink{ref-2011b}{101}}; SIRIUS 4 and
MCnebula were executed in Pop!\_OS (Ubuntu) 20.10 LTS 64-bits PC (Intel
Core i7-1065G7, 1.3 GHz \(\times\) 8, 16 Gb of
RAM)\textsuperscript{\protect\hyperlink{ref-2019b}{129}}. For the
evaluation dataset (noise simulation dataset and serum dataset), all
MCnebula workflow were implemented on Pop!-OS (Ubuntu) 20.04 LTS 64-bits
workstation (Intel Core i9-10900X, 3.70GHz \(\times\) 20, 125.5 Gb of
RAM). R packages and custom R script were extensively used for data
processing and scientific mapping.

\hypertarget{data-and-code-availability}{%
\section{\texorpdfstring{\textbf{Data and code
availability}}{Data and code availability}}\label{data-and-code-availability}}

   The serum dataset were available at MassIVE web service (id no.
\href{https://massive.ucsd.edu/ProteoSAFe/QueryMSV?id=MSV000079949}{MSV000083593}).
The submission job in GNPS of evaluation dataset are available: 1)
original dataset: FBMN:
\url{https://gnps.ucsd.edu/ProteoSAFe/status.jsp?task=05f492249df5413ba72a1def76ca973d}.
MolnetEnhancer:
\url{https://gnps.ucsd.edu/ProteoSAFe/status.jsp?task=9d9c7f83fa2046c2bf615a3dbe35ca62};
2) middle noise dataset: FBMN:
\url{https://gnps.ucsd.edu/ProteoSAFe/status.jsp?task=c65abe76cd9846c99f1ae47ddbd34927};
MolnetEnhancer:
\url{https://gnps.ucsd.edu/ProteoSAFe/status.jsp?task=7cc8b5a2476f4d4e90256ec0a0f94ca7};
3) high noise dataset: FBMN:
\url{https://gnps.ucsd.edu/ProteoSAFe/status.jsp?task=62b25cf2dcf041d3a8b5593fdbf5ac5e};
MolnetEnhancer:
\url{https://gnps.ucsd.edu/ProteoSAFe/status.jsp?task=f6d08a335e814c5eac7c97598b26fb80}.

   The source code of MCnebula integrated in R is available at
\url{https://github.com/Cao-lab-zcmu/MCnebula}. Other R scripts used for
analysis or graphic mapping in this manuscript are available upon
request.

\hypertarget{bibliography}{%
\section*{Referencetttttttt}\label{bibliography}}
\addcontentsline{toc}{section}{Referencetttttttt}

\hypertarget{refs}{}
\begin{CSLReferences}{0}{0}
\leavevmode\vadjust pre{\hypertarget{ref-2020p}{}}%
\CSLLeftMargin{1. }%
\CSLRightInline{Tsugawa H, Ikeda K, Takahashi M, Satoh A, Mori Y, Uchino
H, et al. \href{https://doi.org/10.1038/s41587-020-0531-2}{A lipidome
atlas in {MS-DIAL} 4}. Nature Biotechnology. 2020 Oct;38(10):1159--63. }

\leavevmode\vadjust pre{\hypertarget{ref-2018az}{}}%
\CSLLeftMargin{2. }%
\CSLRightInline{Chong J, Soufan O, Li C, Caraus I, Li S, Bourque G, et
al. \href{https://doi.org/10.1093/nar/gky310}{{MetaboAnalyst} 4.0:
Towards more transparent and integrative metabolomics analysis}. Nucleic
Acids Research. 2018 Jul;46(W1):W486--94. }

\leavevmode\vadjust pre{\hypertarget{ref-2020co}{}}%
\CSLLeftMargin{3. }%
\CSLRightInline{Tsugawa H.
\href{https://doi.org/10.1016/B978-0-12-409547-2.14645-1}{Computational
{MS}/{MS Fragmentation} and {Structure Elucidation Using MS-FINDER
Software}}. In: Comprehensive {Natural Products III}. {Elsevier}; 2020.
p. 189--210. }

\leavevmode\vadjust pre{\hypertarget{ref-2016a}{}}%
\CSLLeftMargin{4. }%
\CSLRightInline{Wang M, Carver JJ, Phelan VV, Sanchez LM, Garg N, Peng
Y, et al. \href{https://doi.org/10.1038/nbt.3597}{Sharing and community
curation of mass spectrometry data with {Global Natural Products Social
Molecular Networking}}. Nature Biotechnology. 2016 Aug;34(8):828--37. }

\leavevmode\vadjust pre{\hypertarget{ref-2012d}{}}%
\CSLLeftMargin{5. }%
\CSLRightInline{Chambers MC, Maclean B, Burke R, Amodei D, Ruderman DL,
Neumann S, et al. \href{https://doi.org/ghh626}{A cross-platform toolkit
for mass spectrometry and proteomics}. Nature Biotechnology. 2012
Oct;30(10):918--20. }

\leavevmode\vadjust pre{\hypertarget{ref-2016e}{}}%
\CSLLeftMargin{6. }%
\CSLRightInline{Röst HL, Sachsenberg T, Aiche S, Bielow C, Weisser H,
Aicheler F, et al. \href{https://doi.org/f82r32}{{OpenMS}: A flexible
open-source software platform for mass spectrometry data analysis}.
Nature Methods. 2016 Sep;13(9):741--8. }

\leavevmode\vadjust pre{\hypertarget{ref-2006a}{}}%
\CSLLeftMargin{7. }%
\CSLRightInline{Smith CA, Want EJ, O'Maille G, Abagyan R, Siuzdak G.
\href{https://doi.org/b58xrd}{{XCMS}: {Processing Mass Spectrometry
Data} for {Metabolite Profiling Using Nonlinear Peak Alignment},
{Matching}, and {Identification}}. Analytical Chemistry. 2006
Feb;78(3):779--87. }

\leavevmode\vadjust pre{\hypertarget{ref-2010}{}}%
\CSLLeftMargin{8. }%
\CSLRightInline{Pluskal T, Castillo S, Villar-Briones A, Orešič M.
\href{https://doi.org/bxbwnj}{{MZmine} 2: {Modular} framework for
processing, visualizing, and analyzing mass spectrometry-based molecular
profile data}. BMC Bioinformatics. 2010 Dec;11(1):395. }

\leavevmode\vadjust pre{\hypertarget{ref-2017f}{}}%
\CSLLeftMargin{9. }%
\CSLRightInline{Myers OD, Sumner SJ, Li S, Barnes S, Du X.
\href{https://doi.org/gbrjtm}{One {Step Forward} for {Reducing False
Positive} and {False Negative Compound Identifications} from {Mass
Spectrometry Metabolomics Data}: {New Algorithms} for {Constructing
Extracted Ion Chromatograms} and {Detecting Chromatographic Peaks}}.
Analytical Chemistry. 2017 Sep;89(17):8696--703. }

\leavevmode\vadjust pre{\hypertarget{ref-2022}{}}%
\CSLLeftMargin{10. }%
\CSLRightInline{Fu J, Zhang Y, Wang Y, Zhang H, Liu J, Tang J, et al.
\href{https://doi.org/10.1038/s41596-021-00636-9}{Optimization of
metabolomic data processing using {NOREVA}}. Nature Protocols. 2022
Jan;17(1):129--51. }

\leavevmode\vadjust pre{\hypertarget{ref-2017ao}{}}%
\CSLLeftMargin{11. }%
\CSLRightInline{Mahieu NG, Patti GJ.
\href{https://doi.org/10.1021/acs.analchem.7b02380}{Systems-{Level
Annotation} of a {Metabolomics Data Set Reduces} 25 000 {Features} to
{Fewer} than 1000 {Unique Metabolites}}. Analytical Chemistry. 2017
Oct;89(19):10397--406. }

\leavevmode\vadjust pre{\hypertarget{ref-2022b}{}}%
\CSLLeftMargin{12. }%
\CSLRightInline{Gloaguen Y, Kirwan JA, Beule D.
\href{https://doi.org/10.1021/acs.analchem.1c02220}{Deep
{Learning-Assisted Peak Curation} for {Large-Scale LC-MS Metabolomics}}.
Analytical Chemistry. 2022 Mar;94(12):4930--7. }

\leavevmode\vadjust pre{\hypertarget{ref-2017i}{}}%
\CSLLeftMargin{13. }%
\CSLRightInline{Neumann U, Genze N, Heider D.
\href{https://doi.org/10.1186/s13040-017-0142-8}{{EFS}: An ensemble
feature selection tool implemented as {R-package} and web-application}.
BioData Mining. 2017 Dec;10(1):21. }

\leavevmode\vadjust pre{\hypertarget{ref-2010p}{}}%
\CSLLeftMargin{14. }%
\CSLRightInline{Horai H, Arita M, Kanaya S, Nihei Y, Ikeda T, Suwa K, et
al. \href{https://doi.org/10.1002/jms.1777}{{MassBank}: A public
repository for sharing mass spectral data for life sciences}. Journal of
mass spectrometry: JMS. 2010 Jul;45(7):703--14. }

\leavevmode\vadjust pre{\hypertarget{ref-2012ac}{}}%
\CSLLeftMargin{15. }%
\CSLRightInline{Sawada Y, Nakabayashi R, Yamada Y, Suzuki M, Sato M,
Sakata A, et al.
\href{https://doi.org/10.1016/j.phytochem.2012.07.007}{{RIKEN} tandem
mass spectral database ({ReSpect}) for phytochemicals: {A}
plant-specific {MS}/{MS-based} data resource and database}.
Phytochemistry. 2012 Oct;82:38--45. }

\leavevmode\vadjust pre{\hypertarget{ref-2015ak}{}}%
\CSLLeftMargin{16. }%
\CSLRightInline{Lei Z, Jing L, Qiu F, Zhang H, Huhman D, Zhou Z, et al.
\href{https://doi.org/10.1021/acs.analchem.5b01559}{Construction of an
{Ultrahigh Pressure Liquid Chromatography-Tandem Mass Spectral Library}
of {Plant Natural Products} and {Comparative Spectral Analyses}}.
Analytical Chemistry. 2015 Jul;87(14):7373--81. }

\leavevmode\vadjust pre{\hypertarget{ref-2020cp}{}}%
\CSLLeftMargin{17. }%
\CSLRightInline{Lee S, Hwang S, Seo M, Shin KB, Kim KH, Park GW, et al.
\href{https://doi.org/10.1016/j.phytochem.2020.112427}{{BMDMS-NP}: {A}
comprehensive {ESI-MS}/{MS} spectral library of natural compounds}.
Phytochemistry. 2020 Sep;177:112427. }

\leavevmode\vadjust pre{\hypertarget{ref-2020cm}{}}%
\CSLLeftMargin{18. }%
\CSLRightInline{Wang M, Jarmusch AK, Vargas F, Aksenov AA, Gauglitz JM,
Weldon K, et al. \href{https://doi.org/10.1038/s41587-019-0375-9}{Mass
spectrometry searches using {MASST}}. Nature Biotechnology. 2020
Jan;38(1):23--6. }

\leavevmode\vadjust pre{\hypertarget{ref-2010c}{}}%
\CSLLeftMargin{19. }%
\CSLRightInline{Wolf S, Schmidt S, Müller-Hannemann M, Neumann S.
\href{https://doi.org/10.1186/1471-2105-11-148}{In silico fragmentation
for computer assisted identification of metabolite mass spectra}. BMC
Bioinformatics. 2010 Dec;11(1):148. }

\leavevmode\vadjust pre{\hypertarget{ref-2015c}{}}%
\CSLLeftMargin{20. }%
\CSLRightInline{Allen F, Greiner R, Wishart D.
\href{https://doi.org/10.1007/s11306-014-0676-4}{Competitive
fragmentation modeling of {ESI-MS}/{MS} spectra for putative metabolite
identification}. Metabolomics. 2015 Feb;11(1):98--110. }

\leavevmode\vadjust pre{\hypertarget{ref-2016am}{}}%
\CSLLeftMargin{21. }%
\CSLRightInline{Ruttkies C, Schymanski EL, Wolf S, Hollender J, Neumann
S. \href{https://doi.org/10.1186/s13321-016-0115-9}{{MetFrag}
relaunched: Incorporating strategies beyond in silico fragmentation}.
Journal of Cheminformatics. 2016;8:3. }

\leavevmode\vadjust pre{\hypertarget{ref-2017aq}{}}%
\CSLLeftMargin{22. }%
\CSLRightInline{Blaženović I, Kind T, Torbašinović H, Obrenović S, Mehta
SS, Tsugawa H, et al.
\href{https://doi.org/10.1186/s13321-017-0219-x}{Comprehensive
comparison of in silico {MS}/{MS} fragmentation tools of the {CASMI}
contest: Database boosting is needed to achieve 93\% accuracy}. Journal
of Cheminformatics. 2017 May;9(1):32. }

\leavevmode\vadjust pre{\hypertarget{ref-2017ap}{}}%
\CSLLeftMargin{23. }%
\CSLRightInline{Wang Y, Wang X, Zeng X.
\href{https://doi.org/10.1007/s11306-017-1258-z}{{MIDAS-G}: A
computational platform for investigating fragmentation rules of tandem
mass spectrometry in metabolomics}. Metabolomics. 2017 Oct;13(10):116. }

\leavevmode\vadjust pre{\hypertarget{ref-2018ax}{}}%
\CSLLeftMargin{24. }%
\CSLRightInline{da Silva RR, Wang M, Nothias L-F, van der Hooft JJJ,
Caraballo-Rodríguez AM, Fox E, et al.
\href{https://doi.org/10.1371/journal.pcbi.1006089}{Propagating
annotations of molecular networks using in silico fragmentation}. PLoS
computational biology. 2018 Apr;14(4):e1006089. }

\leavevmode\vadjust pre{\hypertarget{ref-2019bn}{}}%
\CSLLeftMargin{25. }%
\CSLRightInline{Djoumbou-Feunang Y, Pon A, Karu N, Zheng J, Li C, Arndt
D, et al. \href{https://doi.org/10.3390/metabo9040072}{{CFM-ID} 3.0:
{Significantly Improved ESI-MS}/{MS Prediction} and {Compound
Identification}}. Metabolites. 2019 Apr;9(4):72. }

\leavevmode\vadjust pre{\hypertarget{ref-2020}{}}%
\CSLLeftMargin{26. }%
\CSLRightInline{Ludwig M, Fleischauer M, Dührkop K, Hoffmann MA, Böcker
S. \href{https://doi.org/10.1007/978-1-0716-0239-3_11}{De {Novo
Molecular Formula Annotation} and {Structure Elucidation Using SIRIUS}
4}. In: Li S, editor. Computational {Methods} and {Data Analysis} for
{Metabolomics}. {New York, NY}: {Springer US}; 2020. p. 185--207. }

\leavevmode\vadjust pre{\hypertarget{ref-2020cn}{}}%
\CSLLeftMargin{27. }%
\CSLRightInline{Chao A, Al-Ghoul H, McEachran AD, Balabin I, Transue T,
Cathey T, et al. \href{https://doi.org/10.1007/s00216-019-02351-7}{In
silico {MS}/{MS} spectra for identifying unknowns: A critical
examination using {CFM-ID} algorithms and {ENTACT} mixture samples}.
Analytical and Bioanalytical Chemistry. 2020 Feb;412(6):1303--15. }

\leavevmode\vadjust pre{\hypertarget{ref-2013w}{}}%
\CSLLeftMargin{28. }%
\CSLRightInline{Kind T, Liu K-H, Lee DY, DeFelice B, Meissen JK, Fiehn
O. \href{https://doi.org/10.1038/nmeth.2551}{{LipidBlast} in silico
tandem mass spectrometry database for lipid identification}. Nature
Methods. 2013 Aug;10(8):755--8. }

\leavevmode\vadjust pre{\hypertarget{ref-2015aj}{}}%
\CSLLeftMargin{29. }%
\CSLRightInline{Jeffryes JG, Colastani RL, Elbadawi-Sidhu M, Kind T,
Niehaus TD, Broadbelt LJ, et al.
\href{https://doi.org/10.1186/s13321-015-0087-1}{{MINEs}: Open access
databases of computationally predicted enzyme promiscuity products for
untargeted metabolomics}. Journal of Cheminformatics. 2015;7:44. }

\leavevmode\vadjust pre{\hypertarget{ref-2012ab}{}}%
\CSLLeftMargin{30. }%
\CSLRightInline{Heinonen M, Shen H, Zamboni N, Rousu J.
\href{https://doi.org/10.1093/bioinformatics/bts437}{Metabolite
identification and molecular fingerprint prediction through machine
learning}. Bioinformatics (Oxford, England). 2012 Sep;28(18):2333--41. }

\leavevmode\vadjust pre{\hypertarget{ref-2015a}{}}%
\CSLLeftMargin{31. }%
\CSLRightInline{Dührkop K, Shen H, Meusel M, Rousu J, Böcker S.
\href{https://doi.org/10.1073/pnas.1509788112}{Searching molecular
structure databases with tandem mass spectra using {CSI}:{FingerID}}.
Proceedings of the National Academy of Sciences. 2015
Oct;112(41):12580--5. }

\leavevmode\vadjust pre{\hypertarget{ref-2018ay}{}}%
\CSLLeftMargin{32. }%
\CSLRightInline{Ludwig M, Dührkop K, Böcker S.
\href{https://doi.org/10.1093/bioinformatics/bty245}{Bayesian networks
for mass spectrometric metabolite identification via molecular
fingerprints}. Bioinformatics (Oxford, England). 2018
Jul;34(13):i333--40. }

\leavevmode\vadjust pre{\hypertarget{ref-2019bo}{}}%
\CSLLeftMargin{33. }%
\CSLRightInline{Frainay C, Aros S, Chazalviel M, Garcia T, Vinson F,
Weiss N, et al.
\href{https://doi.org/10.1093/bioinformatics/bty577}{{MetaboRank}:
Network-based recommendation system to interpret and enrich metabolomics
results}. Bioinformatics (Oxford, England). 2019 Jan;35(2):274--83. }

\leavevmode\vadjust pre{\hypertarget{ref-2019bk}{}}%
\CSLLeftMargin{34. }%
\CSLRightInline{Nguyen DH, Nguyen CH, Mamitsuka H.
\href{https://doi.org/10.1093/bioinformatics/btz319}{{ADAPTIVE}:
{leArning DAta-dePendenT}, {concIse} molecular {VEctors} for fast,
accurate metabolite identification from tandem mass spectra}.
Bioinformatics. 2019 Jul;35(14):i164--72. }

\leavevmode\vadjust pre{\hypertarget{ref-2021cy}{}}%
\CSLLeftMargin{35. }%
\CSLRightInline{Cao L, Guler M, Tagirdzhanov A, Lee Y-Y, Gurevich A,
Mohimani H.
\href{https://doi.org/10.1038/s41467-021-23986-0}{{MolDiscovery}:
Learning mass spectrometry fragmentation of small molecules}. Nature
Communications. 2021 Dec;12(1):3718. }

\leavevmode\vadjust pre{\hypertarget{ref-2019}{}}%
\CSLLeftMargin{36. }%
\CSLRightInline{Dührkop K, Fleischauer M, Ludwig M, Aksenov AA, Melnik
AV, Meusel M, et al.
\href{https://doi.org/10.1038/s41592-019-0344-8}{{SIRIUS} 4: A rapid
tool for turning tandem mass spectra into metabolite structure
information}. Nature Methods. 2019 Apr;16(4):299--302. }

\leavevmode\vadjust pre{\hypertarget{ref-2020b}{}}%
\CSLLeftMargin{37. }%
\CSLRightInline{Aron AT, Gentry EC, McPhail KL, Nothias L-F,
Nothias-Esposito M, Bouslimani A, et al.
\href{https://doi.org/10.1038/s41596-020-0317-5}{Reproducible molecular
networking of untargeted mass spectrometry data using {GNPS}}. Nature
Protocols. 2020 Jun;15(6):1954--91. }

\leavevmode\vadjust pre{\hypertarget{ref-2000g}{}}%
\CSLLeftMargin{38. }%
\CSLRightInline{Ashburner M, Ball CA, Blake JA, Botstein D, Butler H,
Cherry JM, et al. \href{https://doi.org/10.1038/75556}{Gene ontology:
Tool for the unification of biology. {The Gene Ontology Consortium}}.
Nature Genetics. 2000 May;25(1):25--9. }

\leavevmode\vadjust pre{\hypertarget{ref-2016}{}}%
\CSLLeftMargin{39. }%
\CSLRightInline{Djoumbou Feunang Y, Eisner R, Knox C, Chepelev L,
Hastings J, Owen G, et al.
\href{https://doi.org/10.1186/s13321-016-0174-y}{{ClassyFire}: Automated
chemical classification with a comprehensive, computable taxonomy}.
Journal of Cheminformatics. 2016 Dec;8(1):61. }

\leavevmode\vadjust pre{\hypertarget{ref-2019bt}{}}%
\CSLLeftMargin{40. }%
\CSLRightInline{Blaženović I, Kind T, Sa MR, Ji J, Vaniya A, Wancewicz
B, et al. \href{https://doi.org/10.1021/acs.analchem.8b04698}{Structure
{Annotation} of {All Mass Spectra} in {Untargeted Metabolomics}.}
Analytical chemistry. 2019 Feb;91(3):2155--62. }

\leavevmode\vadjust pre{\hypertarget{ref-2019br}{}}%
\CSLLeftMargin{41. }%
\CSLRightInline{Ernst M, Kang KB, Caraballo-Rodríguez AM, Nothias L-F,
Wandy J, Chen C, et al.
\href{https://doi.org/10.3390/metabo9070144}{{MolNetEnhancer}: {Enhanced
Molecular Networks} by {Integrating Metabolome Mining} and {Annotation
Tools}.} Metabolites. 2019 Jul;9(7). }

\leavevmode\vadjust pre{\hypertarget{ref-2019bs}{}}%
\CSLLeftMargin{42. }%
\CSLRightInline{Lee J, da Silva RR, Jang HS, Kim HW, Kwon YS, Kim J-H,
et al. \href{https://doi.org/10.1016/j.foodchem.2019.05.099}{In silico
annotation of discriminative markers of three {Zanthoxylum} species
using molecular network derived annotation propagation.} Food chemistry.
2019 Oct;295:368--76. }

\leavevmode\vadjust pre{\hypertarget{ref-2019bq}{}}%
\CSLLeftMargin{43. }%
\CSLRightInline{Sha B, Schymanski EL, Ruttkies C, Cousins IT, Wang Z.
\href{https://doi.org/10.1039/c9em00321e}{Exploring open cheminformatics
approaches for categorizing per- and polyfluoroalkyl substances
({PFASs}).} Environmental science Processes \& impacts. 2019
Nov;21(11):1835--51. }

\leavevmode\vadjust pre{\hypertarget{ref-2020cq}{}}%
\CSLLeftMargin{44. }%
\CSLRightInline{Moreno-Ulloa A, Sicairos Diaz V, Tejeda-Mora JA, Macias
Contreras MI, Castillo FD, Guerrero A, et al.
\href{https://doi.org/10.1128/mSystems.00824-20}{Chemical {Profiling
Provides Insights} into the {Metabolic Machinery} of
{Hydrocarbon-Degrading Deep-Sea Microbes}.} mSystems. 2020 Nov;5(6). }

\leavevmode\vadjust pre{\hypertarget{ref-2021b}{}}%
\CSLLeftMargin{45. }%
\CSLRightInline{Tripathi A, Vázquez-Baeza Y, Gauglitz JM, Wang M,
Dührkop K, Nothias-Esposito M, et al.
\href{https://doi.org/10.1038/s41589-020-00677-3}{Chemically informed
analyses of metabolomics mass spectrometry data with {Qemistree}}.
Nature Chemical Biology. 2021 Feb;17(2):146--51. }

\leavevmode\vadjust pre{\hypertarget{ref-2021da}{}}%
\CSLLeftMargin{46. }%
\CSLRightInline{Wang Y-K, Xiao X-R, Zhou Z-M, Xiao Y, Zhu W-F, Liu H-N,
et al. \href{https://doi.org/10.1007/s00216-021-03201-1}{A strategy
combining solid-phase extraction, multiple mass defect filtering and
molecular networking for rapid structural classification and annotation
of natural products: Characterization of chemical diversity in {Citrus}
aurantium as a case study.} Analytical and bioanalytical chemistry. 2021
May;413(11):2879--91. }

\leavevmode\vadjust pre{\hypertarget{ref-2021cz}{}}%
\CSLLeftMargin{47. }%
\CSLRightInline{Neto FC, Raftery D.
\href{https://doi.org/10.1021/acs.analchem.1c02041}{Expanding {Urinary
Metabolite Annotation} through {Integrated Mass Spectral Similarity
Networking}.} Analytical chemistry. 2021 Sep;93(35):12001--10. }

\leavevmode\vadjust pre{\hypertarget{ref-2022al}{}}%
\CSLLeftMargin{48. }%
\CSLRightInline{van Santen JA, Poynton EF, Iskakova D, McMann E, Alsup
TA, Clark TN, et al. \href{https://doi.org/10.1093/nar/gkab941}{The
{Natural Products Atlas} 2.0: A database of microbially-derived natural
products.} Nucleic acids research. 2022 Jan;50(D1):D1317--23. }

\leavevmode\vadjust pre{\hypertarget{ref-2020cr}{}}%
\CSLLeftMargin{49. }%
\CSLRightInline{Quinn RA, Melnik AV, Vrbanac A, Fu T, Patras KA, Christy
MP, et al. \href{https://doi.org/10.1038/s41586-020-2047-9}{Global
chemical effects of the microbiome include new bile-acid conjugations}.
Nature. 2020 Mar;579(7797):123--9. }

\leavevmode\vadjust pre{\hypertarget{ref-2022am}{}}%
\CSLLeftMargin{50. }%
\CSLRightInline{Hezaveh K, Shinde RS, Klötgen A, Halaby MJ, Lamorte S,
Ciudad MT, et al.
\href{https://doi.org/10.1016/j.immuni.2022.01.006}{Tryptophan-derived
microbial metabolites activate the aryl hydrocarbon receptor in
tumor-associated macrophages to suppress anti-tumor immunity}. Immunity.
2022 Feb;55(2):324--340.e8. }

\leavevmode\vadjust pre{\hypertarget{ref-2020s}{}}%
\CSLLeftMargin{51. }%
\CSLRightInline{Wozniak JM, Mills RH, Olson J, Caldera JR, Sepich-Poore
GD, Carrillo-Terrazas M, et al.
\href{https://doi.org/10.1016/j.cell.2020.07.040}{Mortality {Risk
Profiling} of {Staphylococcus} aureus {Bacteremia} by {Multi-omic Serum
Analysis Reveals Early Predictive} and {Pathogenic Signatures}}. Cell.
2020 Sep;182(5):1311--1327.e14. }

\leavevmode\vadjust pre{\hypertarget{ref-2021a}{}}%
\CSLLeftMargin{52. }%
\CSLRightInline{Dührkop K, Nothias L-F, Fleischauer M, Reher R, Ludwig
M, Hoffmann MA, et al.
\href{https://doi.org/10.1038/s41587-020-0740-8}{Systematic
classification of unknown metabolites using high-resolution
fragmentation mass spectra}. Nature Biotechnology. 2021
Apr;39(4):462--71. }

\leavevmode\vadjust pre{\hypertarget{ref-2009}{}}%
\CSLLeftMargin{53. }%
\CSLRightInline{Böcker S, Letzel MC, Lipták Z, Pervukhin A.
\href{https://doi.org/10.1093/bioinformatics/btn603}{{SIRIUS}:
Decomposing isotope patterns for metabolite identification\textdagger}.
Bioinformatics. 2009 Jan;25(2):218--24. }

\leavevmode\vadjust pre{\hypertarget{ref-2015}{}}%
\CSLLeftMargin{54. }%
\CSLRightInline{Dührkop K, Böcker S.
\href{https://doi.org/10.1007/978-3-319-16706-0_10}{Fragmentation {Trees
Reloaded}}. In: Przytycka TM, editor. Research in {Computational
Molecular Biology}. {Cham}: {Springer International Publishing}; 2015.
p. 65--79. }

\leavevmode\vadjust pre{\hypertarget{ref-2020a}{}}%
\CSLLeftMargin{55. }%
\CSLRightInline{Ludwig M, Nothias L-F, Dührkop K, Koester I, Fleischauer
M, Hoffmann MA, et al.
\href{https://doi.org/10.1038/s42256-020-00234-6}{Database-independent
molecular formula annotation using {Gibbs} sampling through {ZODIAC}}.
Nature Machine Intelligence. 2020 Oct;2(10):629--41. }

\leavevmode\vadjust pre{\hypertarget{ref-2020d}{}}%
\CSLLeftMargin{56. }%
\CSLRightInline{Nothias L-F, Petras D, Schmid R, Dührkop K, Rainer J,
Sarvepalli A, et al.
\href{https://doi.org/10.1038/s41592-020-0933-6}{Feature-based molecular
networking in the {GNPS} analysis environment}. Nature Methods. 2020
Sep;17(9):905--8. }

\leavevmode\vadjust pre{\hypertarget{ref-2021cr}{}}%
\CSLLeftMargin{57. }%
\CSLRightInline{Ferreira L, Morais J, Preto M, Silva R, Urbatzka R,
Vasconcelos V, et al.
\href{https://doi.org/10.3390/md19110633}{Uncovering the {Bioactive
Potential} of a {Cyanobacterial Natural Products Library Aided} by
{Untargeted Metabolomics}.} Marine drugs. 2021 Nov;19(11). }

\leavevmode\vadjust pre{\hypertarget{ref-2021cs}{}}%
\CSLLeftMargin{58. }%
\CSLRightInline{Ramabulana A-T, Petras D, Madala NE, Tugizimana F.
\href{https://doi.org/10.3390/metabo11110763}{Metabolomics and
{Molecular Networking} to {Characterize} the {Chemical Space} of {Four
Momordica Plant Species}.} Metabolites. 2021 Nov;11(11). }

\leavevmode\vadjust pre{\hypertarget{ref-2021cu}{}}%
\CSLLeftMargin{59. }%
\CSLRightInline{Villa-Rodriguez E, Moreno-Ulloa A, Castro-Longoria E,
Parra-Cota FI, de Los Santos-Villalobos S.
\href{https://doi.org/10.1016/j.micres.2021.126826}{Integrated omics
approaches for deciphering antifungal metabolites produced by a novel
{Bacillus} species, {B}. Cabrialesii {TE3}({T}), against the spot blotch
disease of wheat ({Triticum} turgidum {L}. Subsp. durum).}
Microbiological research. 2021 Oct;251:126826. }

\leavevmode\vadjust pre{\hypertarget{ref-2021ct}{}}%
\CSLLeftMargin{60. }%
\CSLRightInline{Neto FC, Raftery D.
\href{https://doi.org/10.1021/acs.analchem.1c02041}{Expanding {Urinary
Metabolite Annotation} through {Integrated Mass Spectral Similarity
Networking}.} Analytical chemistry. 2021 Sep;93(35):12001--10. }

\leavevmode\vadjust pre{\hypertarget{ref-2021cx}{}}%
\CSLLeftMargin{61. }%
\CSLRightInline{Courraud J, Ernst M, Svane Laursen S, Hougaard DM, Cohen
AS. \href{https://doi.org/10.1007/s12031-020-01787-2}{Studying {Autism
Using Untargeted Metabolomics} in {Newborn Screening Samples}.} Journal
of molecular neuroscience : MN. 2021 Jul;71(7):1378--93. }

\leavevmode\vadjust pre{\hypertarget{ref-2021cv}{}}%
\CSLLeftMargin{62. }%
\CSLRightInline{Wang Y-K, Xiao X-R, Zhou Z-M, Xiao Y, Zhu W-F, Liu H-N,
et al. \href{https://doi.org/10.1007/s00216-021-03201-1}{A strategy
combining solid-phase extraction, multiple mass defect filtering and
molecular networking for rapid structural classification and annotation
of natural products: Characterization of chemical diversity in {Citrus}
aurantium as a case study.} Analytical and bioanalytical chemistry. 2021
May;413(11):2879--91. }

\leavevmode\vadjust pre{\hypertarget{ref-2021cw}{}}%
\CSLLeftMargin{63. }%
\CSLRightInline{Brites-Neto J, Maimone NM, Piedade SMDS, Andrino FG,
Andrade PAM de, Baroni F de A, et al.
\href{https://doi.org/10.1016/j.jip.2021.107541}{Scorpionicidal activity
of secondary metabolites from {Paecilomyces} sp. {CMAA1686} against
{Tityus} serrulatus.} Journal of invertebrate pathology. 2021
Feb;179:107541. }

\leavevmode\vadjust pre{\hypertarget{ref-2021cl}{}}%
\CSLLeftMargin{64. }%
\CSLRightInline{Chen X-D, Tang J-J, Feng S, Huang H, Lu F-N, Lu X-M, et
al. \href{https://doi.org/10.2174/1570159X19666210111155110}{Chlorogenic
{Acid Improves PTSD-like Symptoms} and {Associated Mechanisms}.} Current
neuropharmacology. 2021;19(12):2180--7. }

\leavevmode\vadjust pre{\hypertarget{ref-2021}{}}%
\CSLLeftMargin{65. }%
\CSLRightInline{Hoffmann MA, Nothias L-F, Ludwig M, Fleischauer M,
Gentry EC, Witting M, et al.
\href{https://doi.org/10.1038/s41587-021-01045-9}{High-confidence
structural annotation of metabolites absent from spectral libraries}.
Nature Biotechnology. 2021 Oct; }

\leavevmode\vadjust pre{\hypertarget{ref-2018bc}{}}%
\CSLLeftMargin{66. }%
\CSLRightInline{Puskarich MA, Evans CR, Karnovsky A, Das AK, Jones AE,
Stringer KA. \href{https://doi.org/10.1097/SHK.0000000000000997}{Septic
{Shock Nonsurvivors Have Persistently Elevated Acylcarnitines Following
Carnitine Supplementation}}. Shock (Augusta, Ga). 2018 Apr;49(4):412--9.
}

\leavevmode\vadjust pre{\hypertarget{ref-2018bi}{}}%
\CSLLeftMargin{67. }%
\CSLRightInline{Melone MAB, Valentino A, Margarucci S, Galderisi U,
Giordano A, Peluso G.
\href{https://doi.org/10.1038/s41419-018-0313-7}{The carnitine system
and cancer metabolic plasticity}. Cell Death \& Disease. 2018
Feb;9(2):228. }

\leavevmode\vadjust pre{\hypertarget{ref-2003n}{}}%
\CSLLeftMargin{68. }%
\CSLRightInline{Drobnik W, Liebisch G, Audebert F-X, Fröhlich D, Glück
T, Vogel P, et al.
\href{https://doi.org/10.1194/jlr.M200401-JLR200}{Plasma ceramide and
lysophosphatidylcholine inversely correlate with mortality in sepsis
patients}. Journal of Lipid Research. 2003 Apr;44(4):754--61. }

\leavevmode\vadjust pre{\hypertarget{ref-2014ao}{}}%
\CSLLeftMargin{69. }%
\CSLRightInline{Park DW, Kwak DS, Park YY, Chang Y, Huh JW, Lim C-M, et
al. \href{https://doi.org/10.1016/j.jcrc.2014.05.003}{Impact of serial
measurements of lysophosphatidylcholine on 28-day mortality prediction
in patients admitted to the intensive care unit with severe sepsis or
septic shock}. Journal of Critical Care. 2014 Oct;29(5):882.e5--11. }

\leavevmode\vadjust pre{\hypertarget{ref-2017au}{}}%
\CSLLeftMargin{70. }%
\CSLRightInline{Strnad P, Tacke F, Koch A, Trautwein C.
\href{https://doi.org/10.1038/nrgastro.2016.168}{Liver - guardian,
modifier and target of sepsis}. Nature Reviews Gastroenterology \&
Hepatology. 2017 Jan;14(1):55--66. }

\leavevmode\vadjust pre{\hypertarget{ref-2021n}{}}%
\CSLLeftMargin{71. }%
\CSLRightInline{Huang L, Lyu Q, Zheng W, Yang Q, Cao G.
\href{https://doi.org/gnmwxx}{Traditional application and modern
pharmacological research of {Eucommia} ulmoides {Oliv}.} Chinese
Medicine. 2021 Dec;16(1):73. }

\leavevmode\vadjust pre{\hypertarget{ref-2021cq}{}}%
\CSLLeftMargin{72. }%
\CSLRightInline{Huang Q, Tan J-B, Zeng X-C, Wang Y-Q, Zou Z-X, Ouyang
D-S. \href{https://doi.org/10.1080/14786419.2019.1700250}{Lignans and
phenolic constituents from {Eucommia} ulmoides {Oliver}.} Natural
product research. 2021 Oct;35(20):3376--83. }

\leavevmode\vadjust pre{\hypertarget{ref-2021bt}{}}%
\CSLLeftMargin{73. }%
\CSLRightInline{Huang Q, Zhang F, Liu S, Jiang Y, Ouyang D.
\href{https://doi.org/10.1016/j.biopha.2021.111735}{Systematic
investigation of the pharmacological mechanism for renal protection by
the leaves of {Eucommia} ulmoides {Oliver} using {UPLC-Q-TOF}/{MS}
combined with network pharmacology analysis.} Biomedicine \&
pharmacotherapy = Biomedecine \& pharmacotherapie. 2021 Aug;140:111735.
}

\leavevmode\vadjust pre{\hypertarget{ref-2021ch}{}}%
\CSLLeftMargin{74. }%
\CSLRightInline{Chen Y, Pan R, Zhang J, Liang T, Guo J, Sun T, et al.
\href{https://doi.org/10.1016/j.jep.2021.113920}{Pinoresinol diglucoside
({PDG}) attenuates cardiac hypertrophy via {AKT}/{mTOR}/{NF-\(\kappa\)B}
signaling in pressure overload-induced rats.} Journal of
ethnopharmacology. 2021 May;272:113920. }

\leavevmode\vadjust pre{\hypertarget{ref-2021aw}{}}%
\CSLLeftMargin{75. }%
\CSLRightInline{Han R, Yuan T, Yang Z, Zhang Q, Wang W-W, Lin L-B, et
al. \href{https://doi.org/10.1016/j.bioorg.2021.105345}{Ulmoidol, an
unusual nortriterpenoid from {Eucommia} ulmoides {Oliv}. {Leaves}
prevents neuroinflammation by targeting the {PU}.1 transcriptional
signaling pathway.} Bioorganic chemistry. 2021 Nov;116:105345. }

\leavevmode\vadjust pre{\hypertarget{ref-2018u}{}}%
\CSLLeftMargin{76. }%
\CSLRightInline{Zhu Y-L, Sun M-F, Jia X-B, Zhang P-H, Xu Y-D, Zhou Z-L,
et al. \href{https://doi.org/10.1097/WNR.0000000000001075}{Aucubin
alleviates glial cell activation and preserves dopaminergic neurons in
1-methyl-4-phenyl-1,2,3,6-tetrahydropyridine-induced parkinsonian mice.}
Neuroreport. 2018 Sep;29(13):1075--83. }

\leavevmode\vadjust pre{\hypertarget{ref-2016aj}{}}%
\CSLLeftMargin{77. }%
\CSLRightInline{Li Y, Gong Z, Cao X, Wang Y, Wang A, Zheng L, et al.
\href{https://doi.org/10.1007/s13318-015-0282-5}{A {UPLC-MS Method} for
{Simultaneous Determination} of {Geniposidic Acid}, {Two Lignans} and
{Phenolics} in {Rat Plasma} and its {Application} to {Pharmacokinetic
Studies} of {Eucommia} ulmoides {Extract} in {Rats}.} European journal
of drug metabolism and pharmacokinetics. 2016 Oct;41(5):595--603. }

\leavevmode\vadjust pre{\hypertarget{ref-2015q}{}}%
\CSLLeftMargin{78. }%
\CSLRightInline{Jing X, Huang W-H, Tang Y-J, Wang Y-Q, Li H, Tian Y-Y,
et al. \href{https://doi.org/10.1155/2015/987973}{Eucommia ulmoides
{Oliv}. ({Du-Zhong}) {Lignans Inhibit Angiotensin II-Stimulated
Proliferation} by {Affecting P21}, {P27}, and {Bax Expression} in {Rat
Mesangial Cells}.} Evidence-based complementary and alternative medicine
: eCAM. 2015;2015:987973. }

\leavevmode\vadjust pre{\hypertarget{ref-2021cp}{}}%
\CSLLeftMargin{79. }%
\CSLRightInline{Huang W, Ding L, Zhang N, Li W, Koike K, Qiu F.
\href{https://doi.org/10.1080/14786419.2020.1715402}{Flavonoids from
{Eucommia} ulmoides and their in~vitro hepatoprotective activities.}
Natural product research. 2021 Nov;35(21):3584--91. }

\leavevmode\vadjust pre{\hypertarget{ref-2021cg}{}}%
\CSLLeftMargin{80. }%
\CSLRightInline{Peng M-F, Tian S, Song Y-G, Li C-X, Miao M-S, Ren Z, et
al. \href{https://doi.org/10.1016/j.jep.2021.113947}{Effects of total
flavonoids from {Eucommia} ulmoides {Oliv}. Leaves on polycystic ovary
syndrome with insulin resistance model rats induced by letrozole
combined with a high-fat diet.} Journal of ethnopharmacology. 2021
Jun;273:113947. }

\leavevmode\vadjust pre{\hypertarget{ref-2019ai}{}}%
\CSLLeftMargin{81. }%
\CSLRightInline{Wang Y, Tan X, Li S, Yang S.
\href{https://doi.org/10.2147/OTT.S210497}{The total flavonoid of
{Eucommia} ulmoides sensitizes human glioblastoma cells to radiotherapy
via {HIF-\(\alpha\)}/{MMP-2} pathway and activates intrinsic apoptosis
pathway.} OncoTargets and therapy. 2019;12:5515--24. }

\leavevmode\vadjust pre{\hypertarget{ref-2019x}{}}%
\CSLLeftMargin{82. }%
\CSLRightInline{Xiao D, Yuan D, Tan B, Wang J, Liu Y, Tan B.
\href{https://doi.org/10.1155/2019/9719618}{The {Role} of {Nrf2
Signaling Pathway} in {Eucommia} ulmoides {Flavones Regulating Oxidative
Stress} in the {Intestine} of {Piglets}.} Oxidative medicine and
cellular longevity. 2019;2019:9719618. }

\leavevmode\vadjust pre{\hypertarget{ref-2012a}{}}%
\CSLLeftMargin{83. }%
\CSLRightInline{Watrous J, Roach P, Alexandrov T, Heath BS, Yang JY,
Kersten RD, et al. \href{https://doi.org/10.1073/pnas.1203689109}{Mass
spectral molecular networking of living microbial colonies}. Proceedings
of the National Academy of Sciences. 2012 Jun;109(26):E1743--52. }

\leavevmode\vadjust pre{\hypertarget{ref-2017g}{}}%
\CSLLeftMargin{84. }%
\CSLRightInline{Quinn RA, Nothias L-F, Vining O, Meehan M, Esquenazi E,
Dorrestein PC. \href{https://doi.org/f9qzvt}{Molecular {Networking As} a
{Drug Discovery}, {Drug Metabolism}, and {Precision Medicine Strategy}}.
Trends in Pharmacological Sciences. 2017 Feb;38(2):143--54. }

\leavevmode\vadjust pre{\hypertarget{ref-2018d}{}}%
\CSLLeftMargin{85. }%
\CSLRightInline{da Silva RR, Wang M, Nothias L-F, van der Hooft JJJ,
Caraballo-Rodríguez AM, Fox E, et al.
\href{https://doi.org/gdc9cj}{Propagating annotations of molecular
networks using in silico fragmentation}. Schlessinger A, editor. PLOS
Computational Biology. 2018 Apr;14(4):e1006089. }

\leavevmode\vadjust pre{\hypertarget{ref-2020e}{}}%
\CSLLeftMargin{86. }%
\CSLRightInline{Xie H-F, Kong Y-S, Li R-Z, Nothias L-F, Melnik AV, Zhang
H, et al. \href{https://doi.org/10.1021/acs.jafc.0c02983}{Feature-{Based
Molecular Networking Analysis} of the {Metabolites Produced} by
{\emph{In Vitro}} {Solid-State Fermentation Reveals Pathways} for the
{Bioconversion} of {Epigallocatechin Gallate}}. Journal of Agricultural
and Food Chemistry. 2020 Jul;68(30):7995--8007. }

\leavevmode\vadjust pre{\hypertarget{ref-2021d}{}}%
\CSLLeftMargin{87. }%
\CSLRightInline{Schmid R, Petras D, Nothias L-F, Wang M, Aron AT, Jagels
A, et al. \href{https://doi.org/10.1038/s41467-021-23953-9}{Ion identity
molecular networking for mass spectrometry-based metabolomics in the
{GNPS} environment}. Nature Communications. 2021 Dec;12(1):3832. }

\leavevmode\vadjust pre{\hypertarget{ref-2016aq}{}}%
\CSLLeftMargin{88. }%
\CSLRightInline{Wishart DS.
\href{https://doi.org/10.1038/nrd.2016.32}{Emerging applications of
metabolomics in drug discovery and precision medicine}. Nature Reviews
Drug Discovery. 2016 Jul;15(7):473--84. }

\leavevmode\vadjust pre{\hypertarget{ref-2017at}{}}%
\CSLLeftMargin{89. }%
\CSLRightInline{Liu X, Locasale JW.
\href{https://doi.org/10.1016/j.tibs.2017.01.004}{Metabolomics: {A
Primer}}. Trends in Biochemical Sciences. 2017 Apr;42(4):274--84. }

\leavevmode\vadjust pre{\hypertarget{ref-2016ar}{}}%
\CSLLeftMargin{90. }%
\CSLRightInline{Guma M, Tiziani S, Firestein GS.
\href{https://doi.org/10.1038/nrrheum.2016.1}{Metabolomics in rheumatic
diseases: Desperately seeking biomarkers}. Nature Reviews Rheumatology.
2016 May;12(5):269--81. }

\leavevmode\vadjust pre{\hypertarget{ref-2016ao}{}}%
\CSLLeftMargin{91. }%
\CSLRightInline{Johnson CH, Ivanisevic J, Siuzdak G.
\href{https://doi.org/10.1038/nrm.2016.25}{Metabolomics: Beyond
biomarkers and towards mechanisms}. Nature Reviews Molecular Cell
Biology. 2016 Jul;17(7):451--9. }

\leavevmode\vadjust pre{\hypertarget{ref-2019bv}{}}%
\CSLLeftMargin{92. }%
\CSLRightInline{Degenhardt F, Seifert S, Szymczak S.
\href{https://doi.org/10.1093/bib/bbx124}{Evaluation of variable
selection methods for random forests and omics data sets}. Briefings in
Bioinformatics. 2019 Mar;20(2):492--503. }

\leavevmode\vadjust pre{\hypertarget{ref-2021de}{}}%
\CSLLeftMargin{93. }%
\CSLRightInline{Sharma A, Lysenko A, Boroevich KA, Vans E, Tsunoda T.
\href{https://doi.org/10.1093/bib/bbab297}{{DeepFeature}: Feature
selection in nonimage data using convolutional neural network}.
Briefings in Bioinformatics. 2021 Nov;22(6):bbab297. }

\leavevmode\vadjust pre{\hypertarget{ref-2019c}{}}%
\CSLLeftMargin{94. }%
\CSLRightInline{Platten M, Nollen EAA, Röhrig UF, Fallarino F, Opitz CA.
\href{https://doi.org/gfvk74}{Tryptophan metabolism as a common
therapeutic target in cancer, neurodegeneration and beyond}. Nature
Reviews Drug Discovery. 2019 May;18(5):379--401. }

\leavevmode\vadjust pre{\hypertarget{ref-2020cv}{}}%
\CSLLeftMargin{95. }%
\CSLRightInline{Knuplez E, Marsche G.
\href{https://doi.org/10.3390/ijms21124501}{An {Updated Review} of
{Pro-} and {Anti-Inflammatory Properties} of {Plasma
Lysophosphatidylcholines} in the {Vascular System}}. International
Journal of Molecular Sciences. 2020 Jun;21(12):E4501. }

\leavevmode\vadjust pre{\hypertarget{ref-2016at}{}}%
\CSLLeftMargin{96. }%
\CSLRightInline{Krautbauer S, Eisinger K, Wiest R, Liebisch G, Buechler
C. \href{https://doi.org/10.1016/j.prostaglandins.2016.06.001}{Systemic
saturated lysophosphatidylcholine is associated with hepatic function in
patients with liver cirrhosis}. Prostaglandins \& Other Lipid Mediators.
2016 Jul;124:27--33. }

\leavevmode\vadjust pre{\hypertarget{ref-2021dg}{}}%
\CSLLeftMargin{97. }%
\CSLRightInline{Perino A, Demagny H, Velazquez-Villegas L, Schoonjans K.
\href{https://doi.org/10.1152/physrev.00049.2019}{Molecular {Physiology}
of {Bile Acid Signaling} in {Health}, {Disease}, and {Aging}}.
Physiological Reviews. 2021 Apr;101(2):683--731. }

\leavevmode\vadjust pre{\hypertarget{ref-2018bd}{}}%
\CSLLeftMargin{98. }%
\CSLRightInline{Wang B, Tontonoz P.
\href{https://doi.org/10.1038/s41574-018-0037-x}{Liver {X} receptors in
lipid signalling and membrane homeostasis}. Nature Reviews
Endocrinology. 2018 Aug;14(8):452--63. }

\leavevmode\vadjust pre{\hypertarget{ref-2021di}{}}%
\CSLLeftMargin{99. }%
\CSLRightInline{Zhang Q, Yao D, Rao B, Jian L, Chen Y, Hu K, et al.
\href{https://doi.org/10.1038/s41467-021-27244-1}{The structural basis
for the phospholipid remodeling by lysophosphatidylcholine
acyltransferase 3}. Nature Communications. 2021 Nov;12(1):6869. }

\leavevmode\vadjust pre{\hypertarget{ref-2010a}{}}%
\CSLLeftMargin{100. }%
\CSLRightInline{Pluskal T, Castillo S, Villar-Briones A, Orešič M.
\href{https://doi.org/bxbwnj}{{MZmine} 2: {Modular} framework for
processing, visualizing, and analyzing mass spectrometry-based molecular
profile data}. BMC Bioinformatics. 2010 Dec;11(1):395. }

\leavevmode\vadjust pre{\hypertarget{ref-2011b}{}}%
\CSLLeftMargin{101. }%
\CSLRightInline{Martens L, Chambers M, Sturm M, Kessner D, Levander F,
Shofstahl J, et al. \href{https://doi.org/dxkg99}{{mzML}\textemdash a
{Community Standard} for {Mass Spectrometry Data}}. Molecular \&
Cellular Proteomics. 2011 Jan;10(1):R110.000133. }

\leavevmode\vadjust pre{\hypertarget{ref-2020v}{}}%
\CSLLeftMargin{102. }%
\CSLRightInline{Gatto L, Gibb S, Rainer J.
\href{https://doi.org/10.1101/2020.04.29.067868}{{MSnbase}, efficient
and elegant {R-based} processing and visualisation of raw mass
spectrometry data}. {Bioinformatics}; 2020 Apr. }

\leavevmode\vadjust pre{\hypertarget{ref-2017}{}}%
\CSLLeftMargin{103. }%
\CSLRightInline{Scheubert K, Hufsky F, Petras D, Wang M, Nothias L-F,
Dührkop K, et al.
\href{https://doi.org/10.1038/s41467-017-01318-5}{Significance
estimation for large scale metabolomics annotations by spectral
matching}. Nature Communications. 2017 Dec;8(1):1494. }

\leavevmode\vadjust pre{\hypertarget{ref-2006b}{}}%
\CSLLeftMargin{104. }%
\CSLRightInline{Csardi G, Nepusz T. The igraph software package for
complex network research. InterJournal. 2006;Complex Systems:1695. }

\leavevmode\vadjust pre{\hypertarget{ref-2003}{}}%
\CSLLeftMargin{105. }%
\CSLRightInline{Shannon P, Markiel A, Ozier O, Baliga NS, Wang JT,
Ramage D, et al. \href{https://doi.org/b7kgpg}{Cytoscape: {A Software
Environment} for {Integrated Models} of {Biomolecular Interaction
Networks}}. Genome Research. 2003 Nov;13(11):2498--504. }

\leavevmode\vadjust pre{\hypertarget{ref-2016g}{}}%
\CSLLeftMargin{106. }%
\CSLRightInline{Wickham H. Ggplot2: {Elegant} graphics for data
analysis. {Springer-Verlag New York}; 2016. }

\leavevmode\vadjust pre{\hypertarget{ref-2021y}{}}%
\CSLLeftMargin{107. }%
\CSLRightInline{Pedersen TL. Ggraph: {An} implementation of grammar of
graphics for graphs and networks. 2021. }

\leavevmode\vadjust pre{\hypertarget{ref-2018h}{}}%
\CSLLeftMargin{108. }%
\CSLRightInline{Xiao N. Ggsci: {Scientific} journal and sci-fi themed
color palettes for 'Ggplot2'. 2018. }

\leavevmode\vadjust pre{\hypertarget{ref-2017j}{}}%
\CSLLeftMargin{109. }%
\CSLRightInline{Auguie B. {gridExtra}: {Miscellaneous} functions for
"{Grid}" graphics. 2017. }

\leavevmode\vadjust pre{\hypertarget{ref-2019l}{}}%
\CSLLeftMargin{110. }%
\CSLRightInline{Potter S, Murrell P. {grImport2}: {Importing} '{SVG}'
graphics. 2019. }

\leavevmode\vadjust pre{\hypertarget{ref-2021ac}{}}%
\CSLLeftMargin{111. }%
\CSLRightInline{Yu G. Ggimage: {Use Image} in 'Ggplot2'. 2021. }

\leavevmode\vadjust pre{\hypertarget{ref-2022f}{}}%
\CSLLeftMargin{112. }%
\CSLRightInline{Ooms J. Rsvg: {Render SVG} images into {PDF}, {PNG},
(encapsulated) {PostScript}, or bitmap arrays. 2022. }

\leavevmode\vadjust pre{\hypertarget{ref-2022d}{}}%
\CSLLeftMargin{113. }%
\CSLRightInline{Wickham H, Henry L, Pedersen TL, Luciani TJ, Decorde M,
Lise V. Svglite: {An} '{SVG}' graphics device. 2022. }

\leavevmode\vadjust pre{\hypertarget{ref-2020u}{}}%
\CSLLeftMargin{114. }%
\CSLRightInline{Pedersen TL. Tidygraph: {A} tidy {API} for graph
manipulation. 2020. }

\leavevmode\vadjust pre{\hypertarget{ref-2021x}{}}%
\CSLLeftMargin{115. }%
\CSLRightInline{Dowle M, Srinivasan A. Data.table: {Extension} of
`data.frame`. 2021. }

\leavevmode\vadjust pre{\hypertarget{ref-2022c}{}}%
\CSLLeftMargin{116. }%
\CSLRightInline{Wickham H, François R, Henry L, Müller K. Dplyr: {A}
grammar of data manipulation. 2022. }

\leavevmode\vadjust pre{\hypertarget{ref-2021aa}{}}%
\CSLLeftMargin{117. }%
\CSLRightInline{Solymos P, Zawadzki Z. Pbapply: {Adding} progress bar to
'*apply' functions. 2021. }

\leavevmode\vadjust pre{\hypertarget{ref-2007a}{}}%
\CSLLeftMargin{118. }%
\CSLRightInline{Wickham H.
\href{https://doi.org/10.18637/jss.v021.i12}{Reshaping data with the
{reshape} package}. Journal of Statistical Software. 2007;21(12):1--20.
}

\leavevmode\vadjust pre{\hypertarget{ref-2019k}{}}%
\CSLLeftMargin{119. }%
\CSLRightInline{Wickham H. Stringr: {Simple}, consistent wrappers for
common string operations. 2019. }

\leavevmode\vadjust pre{\hypertarget{ref-2022e}{}}%
\CSLLeftMargin{120. }%
\CSLRightInline{Wickham H, Girlich M. Tidyr: {Tidy} messy data. 2022. }

\leavevmode\vadjust pre{\hypertarget{ref-2007j}{}}%
\CSLLeftMargin{121. }%
\CSLRightInline{Guha R. Chemical informatics functionality in {R}.
Journal of Statistical Software. 2007;18(6). }

\leavevmode\vadjust pre{\hypertarget{ref-2022ak}{}}%
\CSLLeftMargin{122. }%
\CSLRightInline{Temple Lang D. {RCurl}: {General} network
({HTTP}/{FTP}/...) Client interface for {R}. 2022. }

\leavevmode\vadjust pre{\hypertarget{ref-2012e}{}}%
\CSLLeftMargin{123. }%
\CSLRightInline{Pletnev I, Erin A, McNaught A, Blinov K, Tchekhovskoi D,
Heller S. \href{https://doi.org/10.1186/1758-2946-4-39}{{InChIKey}
collision resistance: An experimental testing}. Journal of
Cheminformatics. 2012 Dec;4(1):39. }

\leavevmode\vadjust pre{\hypertarget{ref-2021c}{}}%
\CSLLeftMargin{124. }%
\CSLRightInline{Hoffmann MA, Nothias L-F, Ludwig M, Fleischauer M,
Gentry EC, Witting M, et al.
\href{https://doi.org/10.1101/2021.03.18.435634}{Assigning confidence to
structural annotations from mass spectra with {COSMIC}}.
{Bioinformatics}; 2021 Mar. }

\leavevmode\vadjust pre{\hypertarget{ref-2020cx}{}}%
\CSLLeftMargin{125. }%
\CSLRightInline{Pang Z, Chong J, Li S, Xia J.
\href{https://doi.org/10.3390/metabo10050186}{{MetaboAnalystR} 3.0:
{Toward} an optimized workflow for global metabolomics}. Metabolites.
2020; }

\leavevmode\vadjust pre{\hypertarget{ref-2018bj}{}}%
\CSLLeftMargin{126. }%
\CSLRightInline{Picart-Armada S, Fernandez-Albert F, Vinaixa M, Yanes O,
Perera-Lluna A.
\href{https://doi.org/10.1186/s12859-018-2487-5}{{FELLA}: An {R} package
to enrich metabolomics data}. BMC Bioinformatics. 2018;19(1):538. }

\leavevmode\vadjust pre{\hypertarget{ref-ilprints422}{}}%
\CSLLeftMargin{127. }%
\CSLRightInline{Page L, Brin S, Motwani R, Winograd T. The {PageRank}
citation ranking: {Bringing} order to the web. {Stanford InfoLab /
Stanford InfoLab}; 1999 Nov. Report No.: 1999-66. }

\leavevmode\vadjust pre{\hypertarget{ref-2010b}{}}%
\CSLLeftMargin{128. }%
\CSLRightInline{Cao Y. The history of {\emph{E}}{\emph{. Ulmoides}} and
the progress of research in the past 20 years. 2010; }

\leavevmode\vadjust pre{\hypertarget{ref-2019b}{}}%
\CSLLeftMargin{129. }%
\CSLRightInline{Lyu Q, Kuo T-H, Sun C, Chen K, Hsu C-C, Li X.
\href{https://doi.org/10.1016/j.foodchem.2019.01.001}{Comprehensive
structural characterization of phenolics in litchi pulp using tandem
mass spectral molecular networking}. Food Chemistry. 2019 Jun;282:9--17.
}

\end{CSLReferences}
